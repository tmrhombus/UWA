\section{Appendix: Lepton efficiency}
\label{sec:tagandprobe}

Leptons have been exploited to trigger the events in the datasets
for this analysis
and to tag the final state of a $W$ boson decay.
In these steps several quality requirements have been applied on
top of the reconstructed leptons,
as described earlier in this chapter.
Selection criteria are applied as well at reconstruction level,
as described in the previous chapter.
The overall efficiency related to these lepton selection requirements plays a
leading role in the global efficiency with which $W + bb$ events are
selected within the defined kinematical phase space.
This efficiency has been measured with a data driven technique,
generally referred to as ``\textit{Tag\&Probe}'' method.
The \textit{Tag\&Probe} exploits the kinematical properties of the
$Z$ boson decay in order to provide an estimate of selection efficiencies
independent from simulation. A sample of $Z$ boson decays is tagged by selecting
events with one isolated lepton and compatible with tight isolation and
identification requirements. The focus of this \textit{tag} lepton selection
is to avoid any possible bias on the identification of additional
leptons in the event, that would mystify the efficiency measurement.
The actual efficiency measurement is performed on top of the second decay lepton,
called \textit{probe},
of the $Z$ with the following simple relation:
$$ \epsilon = \frac{ N_{passing}^{probes}}{N_{passing}^{probes} + N_{failing}^{probes} } \mathrm{.}$$
Both the samples of probes that pass or fail the selection, and its efficiency
is to be measured, contain a fraction of background events where a $Z$ boson has not
been really produced. The fraction of background to both the samples is subtracted
by means of an extended Maximum-Likelihood fit to the invariant mass
of the tag and probe leptons around the $Z$ mass value.
The signal fraction is parametrized by the convolution of
a $Z$ generator shape and a Gaussian spread function taking into account
the response of the detector, while the background is parametrized with several combinations
of exponential and polynomial functions depending on the kinematical configuration of the
tag and probe lepton pair: the efficiency is measured in fact  as a function of the
probe lepton transverse momentum and pseudorapidity. To a smaller extent,
lepton efficiencies depend on the jet multiplicity and number of pileup interactions
in the event as well; these dependencies have not been taken into account, since they
have been found to be negligible within the
statistical uncertainties on the measured efficiency.
In figure~\ref{fig:tagandprobefit} an example of a \textit{Tag\&Probe} fit
is shown.

\begin{figure}[htb]
        \begin{center}
                \leavevmode
                \includegraphics[width=0.6\textwidth]{figs/trieste_plots/tap}
        \end{center}
        \caption{Example of a \textit{Tag\&Probe} fit. In green and red
          the fit to the invariant mass distribution of the
          tag and probe lepton pairs for the distribution of the probes
          respectively passing and failing the selection criterium of
          which the efficiency is to be measured.
          In blue the final simultaneous fit on the global distribution
          in order to compute the efficiency.}
        \label{fig:tagandprobefit}
\end{figure}

In this particular analysis the efficiency of trigger, reconstruction and
offline selection has been separately measured for electrons and muons both
in data and in Monte Carlo events. From these efficiencies,
a set of Monte Carlo scale factors has been derived as a function of the
lepton transverse momentum and pseudorapidity:
$$SF_{HLT,~reco,~offline~ID/ISO} (p_{T},\eta) = \frac{\epsilon_{HLT,~reco,~offline~ID/ISO}^{data}}{\epsilon_{HLT,~reco,~offline~ID/ISO}^{MC}} \mathrm{.}$$
These scale factors have been used to reweight simulated events in order obtain
a consistent description of lepton selection effeciencies with respect to real data.
