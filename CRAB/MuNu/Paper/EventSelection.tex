\section{Event Selection}
\label{sec:wbbselection}

In this section the $W + bb$ final state is
defined along with the detector level kinematical
phase space in which the final state is selected.
A reconstructed lepton is required to be present
in selected events, with transverse momentum $p_{T} > 30$~GeV
and pseudorapidity $|\eta| < 2.1$. In the particular case of electrons
the pseudorapidity range $1.4442 < |\eta| < 1.566$, corresponding
to the gap between the barrel and the endcaps of the ECAL, is
further excluded. The selected lepton needs to satisfy the set of
Tight ID and Tight Isolation
requirements described in the previous section.
The transverse momentum threshold at $p_{T} > 30$~GeV is imposed by
the trigger selection thresholds at $27$~GeV and $24$~GeV for electrons and
muons respectively and the pseudorapidity range $|\eta| < 2.1$ corresponds to the
region where the SingleMu Trigger operated.
These kinematical cuts have been chosen to be symmetric for the two lepton
species in order to statistically combine the cross
section measurements in the two independent channels.
The selection of a final state with one isolated lepton is meant to tag the
visible part of the decay of a $W$ boson.

The contribution from events where a $Z$ boson, a pair of vector bosons
or a $t\bar{t}$~quark pair is produced has been reduced by rejecting
events with an additional reconstructed lepton. Looser requirements
are applied to the additional leptons to be vetoed: they must have transverse momentum
$p_{T} > 10$~GeV, pseudorapidity $|\eta| < 2.4$ which corresponds to the
acceptance of the tracking system, as well as pass the Loose ID and Loose Isolation
 identification and isolation requirements.
A single lepton final state may be faked as well by multijet events (QCD) where a lepton
has been produced in the hadronization process of a jet and misidentified
as an isolated lepton from the primary vertex.
The rate of this kind of misidentification is very low, but, given the cross
section disparity of several orders of magnitudes in favour of multijet events,
the contribution from QCD is not negligible, and has been estimated using the data driven
technique described in section~\ref{sec:qcd}. 

In order to reduce such a background
contribution, a threshold on the transverse mass $m_{\mathrm{T}}$ of the $W$ boson
at $m_{\mathrm{T}} > 45$~GeV has been applied, where  $m_{\mathrm{T}}$ is defined
in Eq. \ref{eq:mt}. 
The missing transverse momentum in multijet events is small and hence the
$m_{T}$ variable has a good discriminating power against QCD background
processes. 
Events are selected where two PF jets have been clustered with the
anti-$k_t$ algorithm (size parameter $R = 0.5$) in the tracker acceptance
($|\eta| < 2.4$) and with transverse momentum $p_{T} > 25$~GeV.
Reconstructed jets within a distance of $R < 0.5$ to the nearest reconstructed and isolated
lepton are rejected, in order to exclude leptons clustered as jets.
Both the jets are required to satisfy a CSV tight $b$-tagging requirement
(CSV discriminator~$> 0.898$).
Events with additional jets are rejected. The purpose of this extra-jet veto
is to control the amount of contamination from $t\bar{t}$~quark pair processes.
The $t\bar{t}$ production is the most relevant contribution to the background
in the present analysis and it becomes the dominant process in events with an
associated jet multiplicity greater than two.
In addition to these selections, events are rejected if any jet is
reconstructed in the forward region of the detector at pseudorapidity
$|\eta| > 2.4$. Such a forward jet veto is expected to reduce the contribution
from the production of single $t$~quarks: the topology of these events is
characterized by the emission of a forward jet in the $s$-channel production mode.

\subsection{Pileup multiplicity}
\label{sec:pileupreweight}

The number of pileup interactions
per event is strongly dependent on the operating
conditions of the LHC colliding beams of protons.
This contribution can be split into the fraction of collisions
between protons belonging to the colliding bunches (\textit{in-time}
pileup) and between protons belonging to the tails of bunches adjacent
to the colliding ones (\textit{out-of-time} pileup).
During the 2012 acquisition period, an average of $21$ pileup interactions per
event were produced, with a peak of $78$ reconstructed pileup vertices
registered in the 2012D dataset.

The pile-up events distribution in data is calculated by the convolution of the
bunch crossing instantaneous luminosity and the total inelastic pp cross section.
In figure~\ref{fig:pileupreweighting} the distribution of the number of
reconstructed vertices is compared
in the 2012 \texttt{SingleMuon} dataset and in simulation before and after the
pileup reweighting procedure, showing that the simulation achieves an acceptable degree
of agreement with respect to the real data pileup conditions after the
application of the weights.

\begin{figure}[htb]
        \begin{center}
                \leavevmode
                \includegraphics[width=0.49\textwidth]{figs/trieste_plots/muons/h_recoVTX}
                \includegraphics[width=0.49\textwidth]{figs/trieste_plots/muons/w_recoVTX}
        \end{center}
        \caption{The distribution of the number of reconstructed vertices in simulation is
          compared against
          the reconstructed number of vertices in real data (black dots) before and
          after the pileup reweighting procedure, in the left and right plots respectively.}
        \label{fig:pileupreweighting}
\end{figure}

\subsection{Selection efficiency}
\label{sec:genlevelEfficiency}

The efficiency of all the selections applied in this analysis
is estimated in simulation within
the generator level fiducial region defined by:
\begin{itemize}
  \item one electron or muon with transverse momentum $p_{T} > 30$~GeV and
    pseudorapidity $|\eta| < 2.1$;
  \item exactly two $b$-jets with transverse momentum $p_{T} > 25$~GeV and
    pseudorapidity $|\eta| < 2.4$;
\end{itemize}
events with additional leptons or jets are rejected.
The selected leptons energy has been corrected by summing
the vector four-momenta of
all the photons generated within a cone of radius $R = 0.1$
around the lepton. This correction is meant to recover the
energy radiated via electroweak final state radiation
by leptons produced in the hard scattering process.
Additionally, leptons generated from the decay of a $B$~hadron have
been rejected, in order to suppress the contribution from
semileptonic $b$ decays.

Generator level jets are clustered on top of the full list of generated
stable particles in the event with the exception of neutrinos.
Jets with a distance smaller than $R = 0.5$ with respect to a
lepton are rejected.
The definition of a
$b$~jet at generator level consists in the requirement of a $B$~hadron to be
present among the jet constituents and within a cone of radius $R = 0.4$
with respect to the jet axis.
No additional cut is applied on the transverse mass of the $W$ boson.
In this fiducial region, the selection efficiency for this analysis has
been measured to be equal to:
$$\epsilon (W \rightarrow e \nu + bb) = (6.99 \pm 1.14) ~ \% $$
$$\epsilon (W \rightarrow \mu \nu + bb) = (8.20 \pm 1.23) ~ \% $$
The measurement has been performed on top of the sample of signal $W + bb$ events
generated with \textsc{MadGraph} in the $4$-flavour scheme 
and the associated uncertainty corresponds to the propagation of statistical
uncertainties on the sample of generated events.
The theoretical uncertainty related to the particular choice of
the PDF and of the renormalization and factorization scales have not been taken into account here.
%This point wil be discussed again in the next chapter together with the estimate of 
%systematic uncertainties. 
As a reference, the contribution of PDF and
renormalization and factorization scales in the
measurement of the selection efficiency has been
estimated in the previous $W + bb$ analysis performed
in CMS on the sample of proton-proton collisions at $7$~TeV~\cite{Chatrchyan:2013uza} and
has been found to contribute for the $10\%$ of the measured cross section.


%The signal events are selected requiring:
%
%\begin{itemize}
%\item One muon or electron  
% \subitem $\pt^\ell>30~\GeV$ and $\abs{\eta^{\ell}}<2.1$ 
% \subitem passes trigger and Tight ID requirements
% \subitem passes the Tight Isolation requirements, muon(electron): $I_{rel}^{PF}<0.12(0.10)$
%\item Rejection of events with a second lepton
% \subitem $\pt^\ell>10~\GeV$ and $\abs{\eta^\ell}<2.4$
% \subitem which passes Loose ID and isolation, muon(electron):  $I_{rel}^{PF}<0.20(0.15)$
%\item Two jets
% \subitem $\pt^{\text{jets}}>25~\GeV$ and $\abs{\eta^{jets}}<2.4$
% \subitem Loose Jet ID
% \subitem both jets pass the CSV tight working point (bdiscriminator > 0.898)
%\item Rejection of events with more than two jets 
%\item Rejection of events with a jet with $\ET>25\GeV$ and $2.4<|\eta|<5$, to reduce the $\ttbar$ contamination.
%\item The transverse mass of the system is required to be $M_{T}>45~\GeV$.
%\end{itemize}
%
%Figure \ref{fig:preselplots} shows the muons $\pt$ and $\eta$, and 
%$\MT$ distributions  
%before requiring b-tagging. Good agreement is observed between data and simulation. 
%%Figure \ref{fig:CSVbdisc} shows the b-tagging CSV discriminator for the leading and subleading jet. 
%
%\begin{figure}
%    \center
%    \subfloat[]{\label{fig:initialmuonpt}\includegraphics[width=0.3\textwidth]{figs/plots/2j0b_mu_pt_prefit.png}}
%    \subfloat[]{\label{fig:initialmuoneta}\includegraphics[width=0.3\textwidth]{figs/plots/2j0b_mu_eta_prefit.png}}
%    \subfloat[]{\label{fig:initialmt}\includegraphics[width=0.3\textwidth]{figs/plots/2j0b_mt_prefit.png}}
%    \caption{(\ref{fig:initialmuonpt}) Transverse momentum of the muon after selecting only isolated Tight ID Muons 
%with $p_{T}>$30 GeV and (\ref{fig:initialmuoneta}) $\eta$ distribution of the muon.
%(\ref{fig:initialmt}) Transverse Mass of the reconstructed
%W boson while requiring that it be produced in association with two PFchs-Jets each with loose ID,
% $p_{T} >$ 25 GeV, and $\eta<$2.4. %The MET used in this analysis has been calibrated using the recoil of the W boson.
% } 
%    \label{fig:preselplots}
%\end{figure}
%
%
