\section{Selection of W + bb events - TRIESTE}
\label{sec:EventSelection}

%In this chapter the structure of the analysis is described 
%in detail and detector level results are presented.
%The primary datasets of proton-proton collisions at $\sqrt{s} = 8$~TeV
%which have been used for this measurement are described in 
%section~\ref{sec:dataset} and the corresponding samples of simulated events 
%for the signal and background processes in section~\ref{sec:mcsamples}.
%The aim of the offline selection implemented for the present analysis
%is to select events where a $W$ boson decaying to an electron or muon 
%and a neutrino has been produced in association with a pair of jets from 
%$b$~quarks.
%Primarily, a set of identification criteria is applied 
%to the physics objects in final states, leptons and jets, 
%in order to further reduce their misidentification rates at reconstruction level. 
%These identification criteria 
%are described in section~\ref{sec:preselection}.
%The actual definition of the final state and its corresponding kinematic acceptance 
%at detector level are presented in section~\ref{sec:wbbselection}.
%%Along with the detector level kinematical acceptance, this section is dedicated to the 
%%descritpion of the generator level acceptance, or phase space, of the analysis: 
%%its precise definition is of fundamental importance
%%for the cross section measurement and 
%%its comparison with theoretical predictions.
%Residual discrepancies between real data and simulation maybe due to a 
%imperfect description of the lepton reconstruction and 
%identification efficiencies and of the b-tagging efficiency. 
%A set of scale factors is computed with data driven techniques and 
%applied to the simulated events in order to mitigate these discrepancies.
%This procedure is explained in section~\ref{sec:scalefactors} together 
%with the details of the reweighting procedure applied to simulated events in 
%order to reproduce the distribution of the multiplicity of pileup 
%interactions in real data.
%The last two sections,~\ref{sec:backgrounds} and~\ref{sec:detectorlevel},
%are dedicated to the estimation and subtraction of all the 
%non negligible backgrounds to the $W + bb$ events and to the presentation 
%of the distribution of these events for the most interesting 
%physical observables.
%%-------------------

In ths chapter there is the descrption of the analysis perfermed in Trieste.
We should discuss how to merge everything. 
The selection of objects and is redundant, while the section regarding the signal extraction 
is totally different, of course...


%-------------------
\subsection{Primary datasets}
\label{sec:dataset}

The focus of the present analysis is to measure the production cross section 
of $W$ bosons in association with $b$-jet pairs. A $W$ boson is most easily 
identified when decaying to an electron or muon and a neutrino and indeed these are 
the decay channels exploited in this measurement. 
The analysis has been performed on top of the 
\texttt{SingleElectron} and \texttt{SingleMuon} primary datasets. These datasets are a subsample
of the proton-proton collision at the center of mass energy of $8$~TeV 
recorded by CMS during 2012.
In particular, the data stored in the \texttt{SingleElectron} and \texttt{SingleMuon} 
datasets belong to events where at least 
one of the several single lepton HLT trigger path has fired.
These samples of events correspond to an integrated luminosity of about  
$20~\mathrm{fb}^{-1}$: the details of the full integrated luminosity
of both the datasets,
along with the details of the integrated luminosity 
for each subsample, can be found in table~\ref{tab:datasampleTS}.
The so-called 2012A, 2012B, 2012C and 2012D subsamples are associated to the four data 
acquisition periods between the three main LHC technical stops in May, July 
and September 2012. The data taking conditions have evolved through each 
technical stops to a higher instantaneous luminosity and, as a consequence,
to a higher number of pileup interactions.
The peak luminosity of $7\times10^{33}$~cm$^{-2}$s$^{-1}$ 
has been reached during the 2012D data taking period and in this configuration 
an average of $21$ pileup interactions per event were produced.

\begin{table}[htb]
\begin{center}
\footnotesize
%\scriptsize
\begin{tabular}{|l|c|}\hline
  Data sample &  Luminosity \\ \hline
  /SingleMu/Run2012A-22Jan2013-v1/AOD & 889 pb$^{-1}$  \\
  /SingleMu/Run2012B-22Jan2013-v1/AOD & 4422 pb$^{-1}$   \\
  /SingleMu/Run2012C-22Jan2013-v1/AOD & 7137 pb$^{-1}$   \\
  /SingleMu/Run2012D-22Jan2013-v1/AOD & 7318 pb$^{-1}$ \\\hline  \hline
  /SingleElectron/Run2012A-22Jan2013-v1/AOD & 889 pb$^{-1}$  \\
  /SingleElectron/Run2012B-22Jan2013-v1/AOD & 4422 pb$^{-1}$   \\
  /SingleElectron/Run2012C-22Jan2013-v1/AOD & 7137 pb$^{-1}$   \\
  /SingleElectron/Run2012D-22Jan2013-v1/AOD & 7318 pb$^{-1}$ \\\hline  \hline
  
  Total (Certified) &  $\cal{L}$ $= 19.8~{\rm fb}^{-1}$  \\ \hline
\end{tabular}
\end{center}
\caption{Datasets and corresponding luminosities used in the analysis.} 
\label{tab:datasampleTS}
\end{table}


\subsubsection{Trigger selection}

As already mentioned, the \texttt{SingleElectron} and \texttt{SingleMuon} primary datasets
are filled with events triggered by at least one single lepton HLT trigger path.
Several trigger paths have been implemented in order to select events 
containing at least one single isolated lepton and the kinematical and 
identification cuts applied by each path have evolved through 2012 data taking 
in order to accomodate the increasing instantaneous luminosity and keep 
the rate of selected events compatible with the data recording capabilities.
As it will be described in the next paragraphs, 
the leptons selected in this analysis have been matched to 
one particular trigger path, whose efficiency has been measured both in 
simulation and real data and propagated to the cross section measurement. 
The trigger path with the less demanding kinematical and identification
requirements that has been kept unprescaled during the whole data acquisition 
is, in the case of electrons:
\begin{itemize}
  \item HLT$\_$SingleEle27.
\end{itemize}
The cuts applied by the selected trigger path consist in a momentum threshold 
at $p_{T} > 27$~GeV, along with some requirements in terms of isolation of the 
calorimetric deposits associated to the electron candidate. No selection in 
pseudorapidity is applied, the only limit being the calorimeter acceptance ($|\eta|<2.5$).
The selected path in the case of muons is:
\begin{itemize}
  \item HLT$\_$IsoMu24$\_$eta2p1,
\end{itemize}
where the momentum threshold of this path has been set to $p_{T} > 24$~GeV
along with a pseudorapidity cut at $|\eta| < 2.1$ and an isolation 
requirement.
The set of selections applied on top of leptons at trigger level 
has been considered when applying the offline selection criteria described later in 
this chapter. Offline selections have been tuned in order to be tighter both in kinematical 
and identification cuts, with the aim of avoiding any inconsistency between the online 
and offline selection. 
%-------------------


%-------------------
\subsection{Monte Carlo samples}
\label{sec:mcsamples}

The same analysis has been performed on a sample of Monte Carlo generated events which 
have undergone the full simulation of the CMS detector with \textsc{GEANT4}.
The $W + bb$ signal process has been simulated with the 
\textsc{MadGraph~5.11}~\cite{MADGRAPH} generator interfaced with \textsc{Pythia6} 
(\textit{tuneZ2*})~\cite{pythia} for the hadronization of partons in the final state; 
the parton distribution function adopted is \textsc{CTEQ6L}~\cite{cteq}.
Two different samples of events, both generated with the same setup, 
have been used in order to select the $W + bb$ signal process.
The first sample consists in events of  associated 
production of $W$ bosons and hadronic jets ($W + jets$),
with the $W$ bosons decaying to leptons.
This sample is fully inclusive with respect 
to the hadronic part of the generated final states and thus 
has been exploited in order to estimate both the fraction of events 
genuinely corresponding to a $W + bb$ event and the fraction of events 
where at least one of the jets in the final state is a light ($u$, $d$, $s$ or $g$) 
quark or $c$~quark jet misidentified as a $b$~jet by the $b$-tagging
algorithm. 
The fraction of events where a $W$ boson has decayed to a $\tau$ 
lepton has been estimated as well in this dataset.
The exact definition of a $b$~jet at generator level will be given in 
section~\ref{sec:genlevelEfficiency}.
The contribution from $b$~quarks in the final states of this dataset
is computed with the massless $5$~flavour scheme ($5$FS).
The main limitation of the dataset described above is due to 
the fact that the
cross section for the production of a $W + bb$ final state
is expected to be at least three orders of magnitude 
smaller than the inclusive cross section for the production 
%of a $W + jets$~\cite{wplusjets7tev}. As a consequence, 
of a $W + jets$. As a consequence, 
the sample of $W + bb$ events is strongly limited in statistics.
In order to partially overcome this problem,
the inclusive $W + jets$ sample has been enriched with events from exclusive samples
with higher jet multiplicities. The four additional samples 
correspond to the generation, with the same setup, of the following 
four \textit{tree~level} processes: $W + 1 jet$, $W + 2 jets$, 
$W + 3 jets$ and $W + 4 jets$.
The exclusive samples have been summed to the inclusive sample after taking into account the 
integrated luminosity of each sample and its corresponding 
cross section, with the aim of achieving a consistent description of the 
jet multiplicity in $W + jets$ final states. For the sake of simplicity,
the higher jet multiplicities in the original $W + jets$ inclusive sample have 
been neglected, given their very small impact on the statistics of the merged 
sample.

A second configuration has been used through the analysis in order to obtain a 
pure sample of signal $W + bb$ events with better statistics when compared to 
the sample of events selected in the $W + jets$ inclusive sample.
For this purpose, the same generator setup has been exploited except for the fact that 
calculations are based upon the massive $4$-flavour scheme (4FS) in this case.
The sample of $W + bb$ events selected on top of these generated events has 
been checked to provide a description of the interesting kinematical distribution of 
the $W + bb$ system that is consistent within statistical uncertainties to the 
one provided by the calculation in the $5$-flavour scheme.
$W + bb$ events selected from the 5FS sample have been removed from the 
final sample for consistency.
In all the plots presented, unless otherwise specified, the 
$4$FS exclusive sample has been adopted in order to simulate the signal events.

All the physics processes with a final state that may be misidentified as a 
$W + bb$ final state have been included in the generation process:
\begin{itemize}
  \item $t\bar{t}$: it has been generated with the \textsc{MadGraph} generator 
    setup both in the fully leptonic and semi-leptonic decay channels;
  \item single $t$: the contribution from the production of a single $t$~quark 
    or $\bar{t}$~quark has been estimated in the $s$-channel, $t$-channel and 
    $tW^{-}$ ($\bar{t}W^{+}$) production modes with events generated by the 
    \textsc{powheg} generator~\cite{POWHEG} interfaced with \textsc{Pythia6} 
    (\textit{tuneZ2*});
  \item \textit{Drell-Yan} ($Z/\gamma* + jets$): these events have been generated with 
    the \textsc{MadGraph} setup already described;
  \item $WW$, $WZ$ and $ZZ$: the contribution from di-boson events is small when compared 
    to the above processes and has been estimated directly with \textsc{Pythia6}.
\end{itemize}

The Monte Carlo samples used in this analysis, both for the simulation of the signal and 
of the background processes, are summarized in
in table~\ref{tab:mcsamples} along with the corresponding number of 
generated events.
Each sample has been rescaled to the integrated luminosity of the data with a correction 
factor calculated with the following simple prescription:
$$c_{norm} = \frac{\mathcal{L}_{DATA} \times \sigma_{MC}}{N_{MC}} \mathrm{,}$$
with $\sigma_{MC}$ being the cross section for the particular process and $N_{MC}$ the 
corresponding number of generated events.
The theoretical prediction for the cross section $\sigma_{MC}$ of each 
simulated process is reported in table~\ref{tab:mcxsecs}.
These cross sections, used to rescale the generated samples, 
represent the best available estimates for every process and 
have been calculated taking into account higher order 
corrections with respect to the setup used for generating the events.
The only exception to this 
procedure is represented by the $t\bar{t}$ full and semi leptonic samples, 
which have been rescaled to the combination of the latest cross section measurement 
from the CMS and ATLAS experiments at $\sqrt{s}=8$~TeV~\cite{ttbarxsecATLASCMS}.
The experimental cross section, being inclusive with respect to the 
$t\bar{t}$ decay modes, has been rescaled by the 
branching ratio corresponding to each process.

\begin{table}[htb]
\begin{center}
\begin{tabular}{|r|l|l|}
\hline
%\begin{adjustwidth}{-4em}{-4em}
%\bf &\bf & \bf{Reference Cross-Sections} \\ 
\textbf{Process}&\textbf{Generator:}& $N_{events}$ \\ 
\hline
$W + $jets $5$FS   & \textsc{MadGraph}  & 57709905 \\ 
$W + $1 jet $5$FS  & \textsc{MadGraph}  & 23141598 \\
$W + $2 jets $5$FS & \textsc{MadGraph}  & 34044921 \\
$W + $3 jets $5$FS & \textsc{MadGraph}  & 15539503 \\
$W + $4 jets $5$FS & \textsc{MadGraph}  & 13382803 \\
$W+b\bar{b}$ $4$FS & \textsc{MadGraph}  & 20646001 \\
\hline
$t\bar{t}$ Full leptonic & \textsc{MadGraph}  & 24963676 \\
$t\bar{t}$ Semi leptonic        & \textsc{MadGraph}  & 12011428 \\
$t$ t-channel      & \textsc{powheg}    & 3758227  \\ 
$t$ s-channel      & \textsc{powheg}    & 259961   \\
$t$ tW-channel     & \textsc{powheg}    & 493460   \\
$\bar{t}$ t-channel  & \textsc{powheg}  & 1935072  \\
$\bar{t}$ s-channel  & \textsc{powheg}  & 139974   \\
$\bar{t}$ tW-channel & \textsc{powheg}  & 493460   \\
\hline
$Z + jets  $ & \textsc{MadGraph}        & 30459503 \\
$WW     $    & \textsc{Pythia}          & 10000431 \\
$WZ     $    & \textsc{Pythia}          & 10000283 \\
$ZZ     $    & \textsc{Pythia}          & 9799908  \\
\hline
\end{tabular}
%\end{adjustwidth}
\end{center}
\caption{Simulated samples used in the analysis.} 
\label{tab:mcsamples}
\end{table}


\begin{table}[htb]
\begin{center}
\begin{tabular}{|r|l|l|l|}
\hline
%\begin{adjustwidth}{-4em}{-4em}
%\bf &\bf & \bf{Reference Cross-Sections}\\ 
\textbf{Process}&\textbf{$\sigma$[pb]}& $\mathcal{O}(\alpha_{S}^{2})$& Generator \\ 
\hline
$W \rightarrow l\nu$ (incl.) $5$FS&36703           & NNLO   & \textsc{FEWZ} \\ 
$W+b\bar{b}$ $4$FS   &138.9           & NLO    & \textsc{aMC@NLO} \\
\hline                                         
$t\bar{t}$ Full leptonic &240.6$\times$BR & DATA& \textsc{-}\\
$t\bar{t}$ Semi leptonic &240.6$\times$BR & DATA& \textsc{-}\\
$t$ t-channel        &46.4            & NLO    & \textsc{Hathor 2.1}\\ 
$t$ s-channel        &3.97            & NNLO   & \cite{singletopXsec}\\
$t$ tW-channel       &11.1            & NNLO   & \cite{singletopXsec}\\
$\bar{t}$ t-channel  &30.7            & NLO    & \textsc{Hathor 2.1}\\
$\bar{t}$ s-channel  &1.76            & NNLO   & \cite{singletopXsec}\\
$\bar{t}$ tW-channel &11.1            & NNLO   & \cite{singletopXsec}\\
\hline                                           
$Z + jets  $         &3532            & NNLO   & \textsc{FEWZ}\\
$WW     $            &54.838          & NLO    & \textsc{MCFM 6.6}\\
$WZ     $            &33.21           & NLO    & \textsc{MCFM 6.6}\\
$ZZ     $            &8.059           & NLO    & \textsc{MCFM 6.6}\\
\hline
\end{tabular}
%\end{adjustwidth}
\end{center}
\caption{The theoretical cross section for each simulated 
  process in the analysis. As the only exception, the $t\bar{t}$ sample 
  has been rescaled to the combination of the latest ATLAS and CMS 
  measurements at $\sqrt{s}=8$~TeV~\cite{ttbarxsecATLASCMS}.} 
\label{tab:mcxsecs}
\end{table}
%-------------------


%-------------------
\subsection{Offline selection}
\label{sec:preselection}

Here, a set of additional selections performed on the 
reconstructed objects of the $W + bb$ final state in the offline analysis 
is presented.
These offline selections are applied to muons, electrons and jets.
In the case of leptons, a set of selection criteria is already applied at 
trigger level; as described at the beginning of this chapter, isolated 
leptons are in fact exploited in order to trigger the events in the 
datasets used for this analysis. A second set of selections is 
applied at reconstruction level, when looking for certain combinations 
of signals in the subdetectors on top of which initiating the lepton 
reconstruction algorithms.
The guiding principle of selection criteria both at trigger and at 
reconstruction level is to achieve the highest possible efficiency, 
at the price of higher misidentification rates.
In light of this fact, the offline selections are tuned 
depending on the needs of the particular physics analysis in terms of 
background rejection.
The sample of selected $W + bb$ events has a strong contamination 
from background processes and thus the tightest quality requirements have been applied 
during the selection of leptons in the final states, together with the rejection of 
particular event topologies as it will be described in section~\ref{sec:wbbselection}.
Given the non negligible impact of these selections on the 
final yield of events, the measurement of their corresponding efficiencies will 
be discussed in detail in section~\ref{sec:tagandprobe}.

\subsubsection{Muons}
\label{sec:muonisooffline}

The following requirements are applied on top of reconstructed muons in order to 
enhance the purity of the selected sample. The first two items of the list are redundant 
with respect to the selections already applied at trigger and reconstruction level,
and aim at avoiding inconsistencies when estimating selection efficiencies:
\begin{itemize}
\item the muon is required to be a global muon, meaning that it needs to have a 
  track from the inner tracking system associated to a track reconstructed in the 
  external muon system, and it is required to pass all the identification 
  requirements at PF reconstruction level;
\item muon segments must be reconstructed in at least 
  two muon stations;
\item the $\chi^{2}/\mathrm{ndof}$ of the global track must be smaller than $10$;
\item at least one good muon chamber hit must be included in the global muon track fit;
\item the transverse and the longitudinal impact parameter of 
  the inner track, $d_{xy}$ and $d_{z}$, as calculated with respect to 
  the primary vertex, are required to satisfy 
  the cuts: $|d_{xy}| < 0.2$~cm, $|d_{z}| < 0.5$~cm;
\item the inner track of the muon must have more than $5$ associated hits in the 
  silicon strip tracker and at least $1$ hit in the pixel detector.
\end{itemize}
This set of tight quality requirements for muons is labelled as muon 
\textit{TightID}. A second set of looser criteria is defined and 
adopted in order to reject events where more than one lepton has been reconstructed and 
consists in the first two items of the \textit{TightID} prescription.
This second set of selections replicates the trigger and reconstruction level 
selections and is generally referred to as muon \textit{LooseID} and its usage 
in the analysis will be clear in the next paragraphs.

In addition to the set of cuts on identification variables described above,
muons are required to be isolated within the detector.
The presence of tracks or calorimetric deposits next to a reconstructed 
muon, and in general to a lepton, is an indication that the lepton itself 
descends with good probability from the hadronization process of a parton from the 
hard scattering. This is the case, for example, of leptons produced 
in the the semileptonic decay of $B$~hadrons within a $b$~jet.
The isolation of PF muons is defined within a cone of radius
$R = 0.4$ around the muon direction by the following formula:
$$I_{\mathrm{rel}}^{\mathrm{PF}} = \frac{\Sigma p_{T}^{\mathrm{charged}}+\mathrm{max}\left( 0, \Sigma E_{T}^{\gamma} + \Sigma E_{T}^{\mathrm{neutral}} - 0.5\Sigma E_{T}^{\mathrm{charged PU}} \right)}{p_{T}^{muon}}$$
with $\Sigma p_{T}^{\mathrm{charged}}$ the sum of transverse momentum of 
all the charged PF particles within the isolation cone, $\Sigma E_{T}^{\mathrm{\gamma}}$ 
and $\Sigma E_{T}^{neutral}$ the sum of the energy deposited in the calorimeters 
by photons and neutral hadrons. The isolation term related to charged particles is corrected 
from the contribution of pileup interactions thanks to the CHS procedure.
The same assumption does not hold in the case of the terms including 
photons and neutral hadrons contribution to isolation. 
These terms are corrected for pileup effects 
by subtracting the additional term $\Sigma E_{T}^{\mathrm{charged PU}}$.
This term is the sum of the transverse momentum of all the charged PF particles 
from pileup interactions within the isolation cone; 
the contribution of photons and neutral hadrons from pileup to the 
isolation has been estimated to be equivalent to half of the 
contribution of charged particles~\cite{neutralpileupestimate-PAS-PFT-10-002},
hence the $0.5$ multiplicative factor.
This term is generally referred to as ``$\Delta\beta$~correction
factor'' to the PF isolation.
\begin{table}[h]
\begin{center}
\begin{tabular}{rcc}
\hline
& \textit{LooseISO} & \textit{TightISO}\\
\hline
Muon PF isolation ($\Delta\beta$-corr.) $<$ & $0.20$ & $0.12$\\
 \hline
\end{tabular}
\end{center}
\caption{Muon PF based isolation corrected for pileup effects with the 
  $\Delta\beta$-correction with thresholds for loose and tight selection
  prescriptions.}
\label{tab:muoniso}
\end{table}
The values of the cut on the isolation variable for muons are reported in 
table~\ref{tab:muoniso} both for a \textit{TightISO} prescription
to be used to select the decay lepton of the $W$ boson and for a 
\textit{LooseISO} prescription to reject events with additional 
reconstructed leptons.

\subsubsection{Electrons}

The offline selection for PF electrons is performed on three 
different sets of variables concerning the identification of reconstructed 
electrons, their isolation and the rejection of electrons generated from 
the conversion of photons interacting with the detector material.

\smallskip

\textbf{Identification}

The electron identification is performed on the following set of variables 
describing the quality of the reconstructed electron:
\begin{itemize}
  \item $\Delta\eta$ and $\Delta\phi$ quantify the quality of the spatial matching between
    the electron track in the inner tracking system and the associated supercluster
    in the ECAL calorimeter;
  \item $\sigma_{i\eta i\eta}$: the spread of the electron energy deposit along the 
    $\eta$ direction. The spread due to radiative energy loss is concentrated along the 
    $\phi$ direction as a consequence of the magnetic field configuration and thus an 
    isolated electron is expected to have a small energy spread along $\eta$;
  \item H/E: the fraction of energy deposited in the hadron calorimeter and 
    in the electromagnetic calorimeter. The electromagnetic shower produced by an 
    electron is fully absorbed within the ECAL and this ratio is expected to be small;
  \item $d_{0}$ and $d_{Z}$ are the transverse and longitudinal impact parameters
    of the electron track and are exploited to reject electrons not originated from 
    the primary vertex of the event;
  \item $|1/E - 1/p|$: the energy measured in the calorimeter and the momentum 
    measured by the tracking system need to be consistent within certain limits.
\end{itemize}

\smallskip

\textbf{Isolation}

The isolation of PF electrons within the detector shares the same purpose and 
the same problematics as the isolation definition in the case of muons, described in the previous 
paragraph, and its definition is identical 
for both the lepton species.
The electron isolation is defined within a cone of radius $R=0.3$ around the electron 
trajectory.
The $\Delta\beta$~correction term is applied as well 
in order to correct for pileup effects.

\smallskip

\textbf{Conversion rejection}

The rejection of electrons generated from the conversion of photons 
interacting with the detector material is performed by computing the 
probability of the track fit when including the primary vertex in the track itself. 
The probability is expected to be very low for electrons non originating from the 
primary vertex of the event.
In addition, the electron is rejected when the associated track in 
the inner tracking system has a certain number of missing hits 
in the innermost layers of the pixel detector, corresponding to the nearest 
region to the beam interaction point: the tracks of electrons from photon 
conversions have a higher probability to begin in the later tracker layers, 
after a certain amount of material has been crossed. \\

\bigskip

In table~\ref{tab:elecutsbarrel} the identification, isolation and conversion rejection 
variables cuts for PF electrons 
detected in the ECAL barrel are reported, both for a \textit{tight} and a 
\textit{loose} selection prescription. 
As in the offline selection of muons, the \textit{tight} prescription 
is adopted when selecting the decay electron of a $W$ boson, while 
the \textit{loose} selection is exploited to reject events with additional 
reconstructed leptons.
The corresponding thresholds 
for electrons detected in the 
ECAL endcaps are listed in table~\ref{tab:elecutsendcaps}.

\begin{table}[h]
\begin{center}
\begin{tabular}{rcc}
\hline
$|\eta| < 1.479$ & \textit{Loose} & \textit{Tight}\\
\hline
$|\Delta\eta| <$ & 0.007 & 0.004\\
$|\Delta\phi| <$  & 0.15 & 0.03 \\
$\sigma(i\eta i\eta) <$   & 0.01 & 0.01 \\
H/E $<$  & 0.12 & 0.12 \\
$|1/E - 1/p| <$   & 0.05 & 0.05 \\
\hline
$|d_{0}| <$ & 0.02 cm & 0.02 cm \\
$|d_{Z}| <$  & 0.2 cm & 0.1 cm \\
\hline
PF Rel. Isolation $<$ & 0.15 & 0.10 \\
\hline
Vert. fit probability  & $10^{-6}$ & $10^{-6}$\\
Missing hits $\leq$    & 1 & 0\\
 \hline
\end{tabular}
\end{center}
\caption{Identification, isolation and rejection of converted photons
  variables for PF electrons with the corresponding threshold values for 
  the \textit{TightID} and \textit{LooseID} selection criteria in the 
  ECAL barrel ($|\eta| < 1.479$).}
\label{tab:elecutsbarrel}
\end{table}

\begin{table}[h]
\begin{center}
\begin{tabular}{rcc}
\hline
$|\eta| > 1.479$ & \textit{Loose} & \textit{Tight}\\
\hline
$|\Delta\eta| <$ & 0.007 & 0.004\\
$|\Delta\phi| <$  & 0.15 & 0.03 \\
$\sigma(i\eta i\eta) <$   & 0.01 & 0.01 \\
H/E $<$  & 0.12 & 0.12 \\
$|1/E - 1/p| <$   & 0.05 & 0.05 \\
\hline
$|d_{0}| <$ & 0.02 cm & 0.02 cm \\
$|d_{Z}| <$  & 0.2 cm & 0.1 cm \\
\hline
PF Rel. Isolation $<$ & 0.15 & 0.10 \\
\hline
Vert. fit probability  & $10^{-6}$ & $10^{-6}$\\
Missing hits $\leq$    & 1 & 0\\
 \hline
\end{tabular}
\end{center}
\caption{Identification, isolation and rejection of converted photons
  variables for PF electrons with the corresponding threshold values for 
  the \textit{TightID} and \textit{LooseID} selection criteria in the 
  ECAL endcaps ($|\eta| > 1.479$).}
\label{tab:elecutsendcaps}
\end{table}

\subsection{Jets}
A loose set of identification criteria is applied to PF jets as well, 
in order to reject fake jets resulting from the clusterization of detector noise.
The PF algorithm makes this task straightforward and the rejection is performed 
with a set of requirements on the composition of the clustered jets.
In particular, jets are required to be composed of at least two PF candidate 
particles and at least one of them to be a charged particle. 
%-------------------


%-------------------
\subsection{W + bb events}
\label{sec:wbbselection}

In this section the $W + bb$ final state is 
defined along with the detector level kinematical 
phase space in which the final state is selected.
A reconstructed lepton is required to be present 
in selected events, with transverse momentum $p_{T} > 30$~GeV 
and pseudorapidity $|\eta| < 2.1$. In the particular case of electrons 
the pseudorapidity range $1.4442 < |\eta| < 1.566$, corresponding 
to the gap between the barrel and the endcaps of the ECAL, is 
further excluded. The selected lepton needs to satisfy the set of 
tight isolation and identification (\textit{TightID, TightISO}) 
requirements described in the previous section.
The lepton transverse momentum threshold at $p_{T} > 30$~GeV is imposed by 
the trigger selection thresholds at $27$~GeV and $24$~GeV for electrons and 
muons respectively: both the selected electron or muon are in fact 
required to be triggered by the \texttt{SingleElectron} and 
\texttt{SingleMuon} trigger paths (described in section~\ref{sec:dataset})
respectively.
Equivalently, the pseudorapidity range $|\eta| < 2.1$ corresponds to the 
region where the best muon reconstruction performance is achieved.
These kinematical cuts have been chosen to be symmetric for the two lepton 
species in order to statistically combine the cross 
section measurements in the two independent channels.
The selection of a final state with one isolated lepton is meant to tag the 
visible part of the decay of a $W$ boson.

The contribution from events where a $Z$ boson, a pair of vector bosons 
or a $t\bar{t}$~quark pair is produced has been reduced by rejecting 
events with an additional reconstructed lepton. Looser requirements 
are applied to the additional leptons to be vetoed: transverse momentum 
$p_{T} > 10$~GeV; pseudorapidity $|\eta| < 2.4$, corresponding to the 
acceptance of the tracking system; \textit{LooseID, LooseISO} identification 
and isolation requirements.
A single lepton final state may be faked as well by multijet events (QCD) where a lepton 
has been produced in the hadronization process of a jet and misidentified 
as an isolated lepton from the primary vertex.
The rate of this kind of misidentification is very low, but, given the cross 
section disparity of several orders of magnitudes in favour of multijet events,
the contribution from QCD is not negligible, and has been estimated using the data driven 
technique described in section~\ref{sec:qcd}. In order to reduce such a background 
contribution, a threshold on the transverse mass $m_{\mathrm{T}}$ of the $W$ boson
at $m_{\mathrm{T}} > 45$~GeV has been applied. 
The transverse mass is computed as a function of the 
transverse momentum of the lepton and of the neutrino from the decay of the $W$;
the transverse momentum of the neutrino is assumed to be equal to missing transverse 
momentum $\vec{E}_{T}^{miss}$ in the event:
$$ m_{\mathrm{T}}(p_{T}^{lepton}, \vec{E}_{T}^{miss}) = \sqrt{2 p_{T}^{lepton} E_{T}^{miss} (1-\cos \Delta\phi)} \mathrm{,}$$
with $\Delta\phi$ the polar distance between the transverse momenta.
The missing transverse momentum in multijet events is virtually null and hence the 
$m_{T}$ variable has a good discriminating power against QCD background 
processes. More quantitative arguments in favour of the choice of the $m_T$ variable will 
be described in section~\ref{sec:qcd}.

Events are selected where two PF jets have been clustered with the 
anti-$k_t$ algorithm (size parameter $R = 0.5$) in the tracker acceptance 
($|\eta| < 2.4$) and with transverse momentum $p_{T} > 25$~GeV.
Reconstructed jets with a distance $R < 0.5$ with respect to a reconstructed and isolated  
lepton are rejected, in order to exclude leptons clustered as jets.
Both the jets are required to satisfy a CSV tight $b$-tagging requirement
(CSV discriminator~$> 0.898$). 
Events with additional jets are rejected. The purpose of this extra-jet veto 
is to control the amount of contamination from $t\bar{t}$~quark pair processes.
The $t\bar{t}$ production is the most relevant contribution to the background 
in the present analysis and it becomes the dominant process in events with an 
associated jet multiplicity greater the $2$. 
A dedicated study in a separate control region for the $t\bar{t}$ process 
will be described in section~\ref{sec:ttbar}.
In addition to these selections, events are rejected if any jet is 
reconstructed in the forward region of the detector at pseudorapidity 
$|\eta| > 2.4$. Such a forward jet veto is expected to reduce the contribution 
from the production of single $t$~quarks: the topology of these events is 
characterized by the emission of a forward jet in the $s$-channel production mode.
%-------------------


%-------------------
\subsection{Monte Carlo reweighting and selection efficiency}
\label{sec:scalefactors}
%\markboth{\MakeUppercase{\chaptername\ \thechapter. \ The $W+bb$ process}}{}

All the known sources of discrepancy between data and Monte Carlo simulation 
have been taken into account in this section, and the 
measurement of a set of ad-hoc weights or scaling factors to be applied on an 
event-by-event basis 
will be described with the aim of
obtaining a consistent description 
of real data in simulation.

\subsubsection{Pileup multiplicity}
\label{sec:pileupreweight}

The number of pileup interactions 
per event is strongly dependent on the operating 
conditions of the LHC colliding beams of protons.
This contribution can be split into the fraction of collisions 
between protons belonging to the colliding bunches (\textit{in-time} 
pileup) and between protons belonging to the tails of bunches adjacent 
to the colliding ones (\textit{out-of-time} pileup).
During the 2012 acquisition period, an average of $21$ pileup interactions per 
event were produced, with a peak of $78$ reconstructed pileup vertices 
registered in the 2012D dataset.
A graphical reconstruction of such a harsh pileup condition is represented in 
picture~\ref{fig:highpileup}.

\begin{figure}[htb]
	\begin{center}
		\leavevmode
		\includegraphics[width=0.6\textwidth]{figs/trieste_plots/highpileup}
	\end{center}
	\caption{Graphical reconstruction of an event with a record number 
          of $78$ reconstructed interaction vertices.}
	\label{fig:highpileup}
\end{figure}

The pileup regime has evolved during the data taking in a way that was not known 
a priori with sufficient detail. 
As a consequence, the simulated events have been generated with a 
distribution of the number of pileup interactions that can be easily reweighted in
order to obtain the actual pileup distribution in real data.
In order to achieve this result, an event-by-event weight has been calculated, starting from
the minimum bias cross section together with the instantaneous luminosity for each bunch crossing,
as the ratio between the observed number of pile-up events in data and in simulation:
%the ratio between the total number of events in data with a certain number $i$ 
%of reconstructed vertices and the corresponding number of events in MC with 
%the same number of generated pileup interactions:
$$ w_{PU}(i) = \frac{N_{data}(i)}{N_{MC}(i) \mathrm{.}}$$
The pile-up events distribution in data is calculated by the convolution of the 
bunch crossing instantaneous luminosity and the total inelastic pp cross section.
In figure~\ref{fig:pileupreweighting} the distribution of the number of 
reconstructed vertices is compared 
in the 2012 \texttt{SingleMuon} dataset and in simulation before and after the 
pileup reweighting procedure, showing that the simulation achieves an acceptable degree
of agreement with respect to the real data pileup conditions after the 
application of the weights.

\begin{figure}[htb]
	\begin{center}
		\leavevmode
		\includegraphics[width=0.49\textwidth]{figs/trieste_plots/muons/h_recoVTX}
		\includegraphics[width=0.49\textwidth]{figs/trieste_plots/muons/w_recoVTX}
	\end{center}
	\caption{The distribution of the number of reconstructed vertices in simulation is 
          compared against 
          the reconstructed number of vertices in real data (black dots) before and 
          after the pileup reweighting procedure, in the left and right plots respectively.}
	\label{fig:pileupreweighting}
\end{figure}

\subsubsection{Lepton efficiency}
\label{sec:tagandprobe}

Leptons have been exploited to trigger the events in the datasets 
for this analysis 
and to tag the final state of a $W$ boson decay. 
In these steps several quality requirements have been applied on 
top of the reconstructed leptons, 
as described earlier in this chapter. 
Selection criteria are applied as well at reconstruction level, 
as described in the previous chapter.
The overall efficiency related to these lepton selection requirements plays a 
leading role in the global efficiency with which $W + bb$ events are
selected within the defined kinematical phase space.
This efficiency has been measured with a data driven technique, 
generally referred to as ``\textit{Tag\&Probe}'' method.
The \textit{Tag\&Probe} exploits the kinematical properties of the 
$Z$ boson decay in order to provide an estimate of selection efficiencies 
independent from simulation. A sample of $Z$ boson decays is tagged by selecting
events with one isolated lepton and compatible with tight isolation and 
identification requirements. The focus of this \textit{tag} lepton selection 
is to avoid any possible bias on the identification of additional 
leptons in the event, that would mystify the efficiency measurement.
The actual efficiency measurement is performed on top of the second decay lepton,
called \textit{probe}, 
of the $Z$ with the following simple relation:
$$ \epsilon = \frac{ N_{passing}^{probes}}{N_{passing}^{probes} + N_{failing}^{probes} } \mathrm{.}$$
Both the samples of probes that pass or fail the selection, and its efficiency 
is to be measured, contain a fraction of background events where a $Z$ boson has not 
been really produced. The fraction of background to both the samples is subtracted 
by means of an extended Maximum-Likelihood fit to the invariant mass 
of the tag and probe leptons around the $Z$ mass value. 
The signal fraction is parametrized by the convolution of 
a $Z$ generator shape and a Gaussian spread function taking into account 
the response of the detector, while the background is parametrized with several combinations 
of exponential and polynomial functions depending on the kinematical configuration of the 
tag and probe lepton pair: the efficiency is measured in fact  as a function of the 
probe lepton transverse momentum and pseudorapidity. To a smaller extent, 
lepton efficiencies depend on the jet multiplicity and number of pileup interactions 
in the event as well; these dependencies have not been taken into account, since they  
have been found to be negligible within the 
statistical uncertainties on the measured efficiency.
In figure~\ref{fig:tagandprobefit} an example of a \textit{Tag\&Probe} fit 
is shown.

\begin{figure}[htb]
	\begin{center}
		\leavevmode
		\includegraphics[width=0.6\textwidth]{figs/trieste_plots/tap}
	\end{center}
	\caption{Example of a \textit{Tag\&Probe} fit. In green and red 
          the fit to the invariant mass distribution of the 
          tag and probe lepton pairs for the distribution of the probes 
          respectively passing and failing the selection criterium of 
          which the efficiency is to be measured.
          In blue the final simultaneous fit on the global distribution 
          in order to compute the efficiency.}
	\label{fig:tagandprobefit}
\end{figure}

In this particular analysis the efficiency of trigger, reconstruction and 
offline selection has been separately measured for electrons and muons both 
in data and in Monte Carlo events. From these efficiencies, 
a set of Monte Carlo scale factors has been derived as a function of the 
lepton transverse momentum and pseudorapidity:
$$SF_{HLT,~reco,~offline~ID/ISO} (p_{T},\eta) = \frac{\epsilon_{HLT,~reco,~offline~ID/ISO}^{data}}{\epsilon_{HLT,~reco,~offline~ID/ISO}^{MC}} \mathrm{.}$$
These scale factors have been used to reweight simulated events in order obtain 
a consistent description of lepton selection effeciencies with respect to real data.

\subsubsection{b-tagging efficiency}

The $b$-tagging efficiency is the second most important ingredient to the 
selection efficiency of the present analysis, after the lepton selection 
efficiencies described in the previous section.
%The actual measurement and calculation of a set of Monte Carlo scale factors 
%to make the jet $b$-tagging efficiency in simulation consistent with data has been 
%described in detail in section~\ref{sec:btagefficiency}.
The usage of these scale factor is strongly dependent on the number of jets 
in final state and the $b$-tagging requirements applied on top of these jets.
In this analysis two jets are selected and both of them are required to pass 
$b$-tagging requirements. After carrying out the combinatory logic for this 
configuration, the correct Monte Carlo weight is found to be described by the 
following formula:
$$ w(2|2) = SF_{b,\mathrm{light}}(\mathrm{1st~jet}) \times SF_{b,\mathrm{light}}(\mathrm{2nd~jet}) \mathrm{.}$$
Since these scale factors are applied to Monte Carlo events, the choice 
between the application of the $SF_{b}$ and $SF_{\mathrm{light}}$ scale 
factor relies on generator level information: as it will be explained later 
in the next section, $b$-jets at generator level are defined as jets containing
at least one $B$~hadron among the jet constituents within a distance of $R=0.4$ 
from the axis; when a jet is $b$-tagged at generator level, $SF_{b}$ is used 
in place of $SF_{\mathrm{light}}$.

\subsubsection{Selection efficiency}
\label{sec:genlevelEfficiency}

The efficiency of all the selections applied in this analysis
is estimated in simulation within 
the generator level fiducial region defined by:
\begin{itemize}
  \item one electron or muon with transverse momentum $p_{T} > 30$~GeV and 
    pseudorapidity $|\eta| < 2.1$;
  \item exactly two $b$-jets with transverse momentum $p_{T} > 25$~GeV and 
    pseudorapidity $|\eta| < 2.4$;
\end{itemize}
events with additional leptons or jets are rejected.
The selected leptons energy has been corrected by summing 
the vector four-momenta of 
all the photons generated within a cone of radius $R = 0.1$
around the lepton. This correction is meant to recover the 
energy radiated via electroweak final state radiation 
by leptons produced in the hard scattering process.
Additionally, leptons generated from the decay of a $B$~hadron have 
been rejected, in order to suppress the contribution from 
semileptonic $b$ decays.

Generator level jets are clustered on top of the full list of generated 
stable particles in the event with the exception of neutrinos.
Jets with a distance smaller than $R = 0.5$ with respect to a 
lepton are rejected.
The definition of a 
$b$~jet at generator level consists in the requirement of a $B$~hadron to be 
present among the jet constituents and within a cone of radius $R = 0.4$ 
with respect to the jet axis.
No additional cut is applied on the transverse mass of the $W$ boson.
In this fiducial region, the selection efficiency for this analysis has 
been measured to be equal to:
$$\epsilon (W \rightarrow e \nu + bb) = (6.99 \pm 1.14) ~ \% $$
$$\epsilon (W \rightarrow \mu \nu + bb) = (8.20 \pm 1.23) ~ \% $$
The measurement has been performed on top of the sample of signal $W + bb$ events 
generated with \textsc{MadGraph} in the $4$-flavour scheme (see section~\ref{sec:mcsamples}
for details) and the associated uncertainty corresponds to the propagation of statistical 
uncertainties on the sample of generated events.
The theoretical uncertainty related to the particular choice of 
the PDF and of the renormalization and factorization scales have not been taken into account here. 
%This point wil be discussed again in the next chapter together with the estimate of 
%systematic uncertainties. 
As a reference, the contribution of PDF and 
renormalization and factorization scales in the 
measurement of the selection efficiency has been 
estimated in the previous $W + bb$ analysis performed 
in CMS on the sample of proton-proton collisions at $7$~TeV~\cite{cmsWbb7tev} and 
has been found to contribute for the $10\%$ of the measured cross section.
%-------------------


%-------------------
\subsection{Background estimation}
\label{sec:backgrounds}

In figure~\ref{fig:transvMassPrefit} the transverse mass distribution for events 
selected with the prescription described in section~\ref{sec:wbbselection} both 
in the electron and the muon decay channels of the $W$ boson is shown.
The black dots indicate the events from real data samples. The colored areas indicate 
the contribution from simulated signal (light dashed yellow) and 
background events, as described in section~\ref{sec:mcsamples}.
The simulated events have been reweighted with the set of Monte Carlo scale factors 
described in the previous paragraphs.
The numerical yields of each contribution are reported in table~\ref{tab:yieldsPrefit}.

\begin{figure}[htb]
	\begin{center}
		\leavevmode
		\includegraphics[width=0.8\textwidth]{figs/trieste_plots/electrons/w_mt_wenu_bb_wide}
		\includegraphics[width=0.8\textwidth]{figs/trieste_plots/muons/w_mt_wmnu_bb_wide}
	\end{center}
	\caption{The transverse mass distribution of events in the signal region for the 
        electron (top) and muon (bottom) decay channels of the $W$ bosons.}
	\label{fig:transvMassPrefit}
\end{figure}


\begin{table}[htb]
\begin{center}
\begin{tabular}{|c|c|c|}
\hline
\textbf{Sample} & $W \rightarrow e \nu + bb$ & $W \rightarrow \mu \nu + bb$ \\ 
\hline
DATA        & 5073   & 5322 \\
\hline             
$W + bb$    & 729.3  & 872.0 \\
$W + c$     & 68.6   & 74.0 \\
$W + light$ & 134.5  & 138.1 \\
$W(\rightarrow \tau \nu) + bb$  & 15.4 & 22.6 \\
$t\bar{t}$  & 2028.6 & 2295.9 \\
Single $t$  & 584.9  & 701.2 \\
$WW$        & 92.9   & 113.4 \\
%$WZ$        & - & - \\
%$ZZ$        & - & - \\
$Z + jets$  & 84.2   & 129.4 \\
\hline
\end{tabular}
\end{center}
\caption{Selected number of events with the selections and event-by-event scale factors
  described in sections~\ref{sec:wbbselection} and~\ref{sec:scalefactors}.} 
\label{tab:yieldsPrefit}
\end{table}


\subsubsection{QCD estimation}
\label{sec:qcd}

The contribution from QCD events, shown in green in figure~\ref{fig:transvMassPrefit},
has been modelled in a QCD enriched sample of events in real data.
Such a sample has been obtained by performing on the \texttt{SingleElectron} and 
\texttt{SingleMuon} datasets a set of selections in every aspect identical to the 
one described for signal events, with the following exceptions:
\begin{itemize}
  \item an anti-isolation requirement is applied when selecting leptons.
    PF isolation is required to be $I_{rel} > 0.15$ in the case of electrons 
    and $I_{rel} > 0.20$ in the case of muons, in place of the standard isolation 
    cuts described in tables~\ref{tab:muoniso}, \ref{tab:elecutsbarrel} and~\ref{tab:elecutsendcaps};
  \item no matching between selected leptons and single lepton trigger 
    paths is required; trigger selection includes isolation requirements that 
    would spoil the effect of the previous anti-isolation cut.
\end{itemize}
The transverse mass distribution of the sample of events selected with such requirements
in the case of electrons
is shown in figure~\ref{fig:qcdprefit}, where the black dots represent the data.


\begin{figure}[htb]
	\begin{center}
		\leavevmode
		\includegraphics[width=0.6\textwidth]{figs/trieste_plots/electronsQCD/w_mt_wenu_bb_wide}
	\end{center}
	\caption{QCD enriched sample of events selected by requiring one anti-isoleted electron.
        The black dots represent the data, the colored areas represent the contributions 
        from simulated signal ($W + bb$) and background events.}
	\label{fig:qcdprefit}
\end{figure}

The same selection has been performed on all the simulated processes 
and is represented in the same plot with the same color conventions as in the 
signal region distribution: the only visible contribution here 
comes from $t\bar{t}$ processes. All the simulated contributions from 
the $W + bb$ signal events 
and from the background processes has been subtracted to this sample and the resulting 
shape has been taken as a template in order to estimate the QCD contribution in the 
signal region.
The normalization of the multijet sample in the signal region has been initialized to 
a reasonable value by performing a fit of the shape
on a sideband of the transverse mass distribution ($m_{T} < 20$ GeV) in the signal
region. The outcome of this fit is shown in figure~\ref{fig:qcdpostfit}.

\begin{figure}[htb]
	\begin{center}
		\leavevmode
		\includegraphics[width=0.6\textwidth]{figs/trieste_plots/electronsQCD/w_mt_wenu_bb_wide_doFit}
	\end{center}
	\caption{The shape of the multijet sample obtained in the dedicated control region is fitted 
          to the sideband ($m_{T} < 20$~GeV) of the transverse mass distribution in the signal region.
          Indeed, the contribution of QCD events is estimated with during the final 
          signal extraction procedure described in section~\ref{sec:transversemassfit}.}
	\label{fig:qcdpostfit}
\end{figure}

The QCD contribution has been normalized to the output of this fit on the 
plot shown in figure~\ref{fig:transvMassPrefit}.
The result of such a fit is not to be considered completely reliable 
because of the very limited statistics and background contamination,
hence the precise estimate of the QCD contribution has been delegated 
to a more sophisticated fit procedure (see section~\ref{sec:transversemassfit}) 
on the full transverse mass distribution 
meant to estimate the signal component for the cross section measurement.

\subsubsection{Top-antitop control region}
\label{sec:ttbar}

The production of $t\bar{t}$ pairs is by far the predominant background 
in the selection of $W + bb$ events and hence a dedicate study in two separate 
control regions has been dedicated to its contribution.
The first control region has been obtained from the signal selection 
defined in section~\ref{sec:wbbselection} by adding the requirement of an 
additional lepton with opposite flavour:
events are selected when one electron and one muon are identified with 
transverse momentum $p_{T}>30$~GeV, pseudorapidity $|\eta|<2.1$ and tight identification 
and isolation requirements.
Events with additional leptons are rejected with the same criteria previously described.
Jet selection is unchanged.
This ``$e/\mu$'' selection is particularly efficient with respect to $t\bar{t}$ 
fully-leptonic decays and the transverse mass distribution 
of the selected events is represented in figure~\ref{fig:ttbarFitemu}.
The $t\bar{t}$ component, by far the dominant one, has been fitted to the data and a global scale 
parameter has been extracted from the fit procedure with the following results:
$$c_{t\bar{t}}^{e/\mu}(W \rightarrow e \nu + bb) = 1.044 \pm 0.31 \mathrm{,}$$
$$c_{t\bar{t}}^{e/\mu}(W \rightarrow \mu \nu + bb) = 1.056 \pm 0.031 \mathrm{,}$$
in the \texttt{SingleElectron} and \texttt{SingleMuon} primary datasets, 
respectively, where the associated uncertainty corresponds to 
the statistical uncertainty of the fit.
The results of the fit are consistent in the two statistically independent 
samples and indicate an excess of data events with respect to the simulation. 
The disagreement is nevertheless compatible with the 
$t\bar{t}$ measured cross section,
to which the pre-fit sample is normalized, within its corresponding uncertainty 
of $3.5\%$ (see section~\ref{sec:mcsamples} for details).

A second control region has been defined with an alternative selection 
on the multiplicity of jets: events are selected when containing at least $3$ jets 
with transverse momentum $p_{T}>25$~GeV and pseudorapidity $|\eta|<2.4$. 
Events are rejected when additional jets with transverse momentum $p_{T}>25$~GeV 
have been clustered in the forward region of the detector ($2.4<|\eta|<5.0$).
The selection applied to leptons is unchanged with respect to the signal selection 
(sec.~\ref{sec:wbbselection}). The strong predominance of $t\bar{t}$ events at higher 
jet multiplicities is especially evident when looking at the fraction of simulated 
samples as a function of the jet multiplicity in figure~\ref{fig:jetmultiplicity}:
this second control sample is more efficient than respect to $t\bar{t}$ 
semi-leptonic decays. As a consequence of the different selection,
it is fully uncorrelated from the previous $e/\mu$ control sample.
The sample of selected events with this ``multi-jet'' 
selection is reported in figure~\ref{fig:ttbarFitmultijet}.
The scaling factors extracted from the fit of the simulated $t\bar{t}$ sample 
to the data are the following:
$$c_{t\bar{t}}^{\mathrm{multi-jet}}(W \rightarrow e \nu + bb) = 1.141 \pm 0.016 \mathrm{,}$$
$$c_{t\bar{t}}^{\mathrm{multi-jet}}(W \rightarrow \mu \nu + bb) =1.118 \pm 0.013 \mathrm{,}$$
and the excess of data is stronger in the multi-jet control sample. 
Still, experimental uncertainties have to be taken into account beside the 
$3.5\%$ $t\bar{t}$ cross section uncertainty and their contribution 
will be described in the next chapter.
For example, the systematic uncertainty due to the $b$-tagging data/MC scale factors alone 
is of the order of $6\%$ and may partially explain such a disagreement.


\begin{figure}[htb]
	\begin{center}
		\leavevmode
		\includegraphics[width=0.49\textwidth]{figs/trieste_plots/electronsTOP.old/w_mt_wenu_bb_wide_doFit}
		\includegraphics[width=0.49\textwidth]{figs/trieste_plots/muonsTOP.old/w_mt_wmnu_bb_wide_doFit}
	\end{center}
	\caption{Sample of data and Monte Carlo events selected in the $t\bar{t}$ ``$e/\mu$'' control region. 
          The results in the electron (left) and muon (right) decay channel are reported.}
	\label{fig:ttbarFitemu}
\end{figure}

\begin{figure}[htb]
	\begin{center}
		\leavevmode
		\includegraphics[width=0.49\textwidth]{figs/trieste_plots/electronsTOP/w_jetmultiplicity}
		\includegraphics[width=0.49\textwidth]{figs/trieste_plots/muonsTOP/w_jetmultiplicity}
	\end{center}
	\caption{The multiplicity of jets in the sample 
          of events selected with the $t\bar{t}$ multi-jet selection 
          criteria, described in~\ref{sec:ttbar}.}
	\label{fig:jetmultiplicity}
\end{figure}

\begin{figure}[htb]
	\begin{center}
		\leavevmode
		\includegraphics[width=0.49\textwidth]{figs/trieste_plots/electronsTOP/w_mt_wenu_bb_wide_doFit}
		\includegraphics[width=0.49\textwidth]{figs/trieste_plots/muonsTOP/w_mt_wmnu_bb_wide_doFit}
	\end{center}
	\caption{Sample of data and Monte Carlo events selected in the $t\bar{t}$ ``multi-jet'' control region. 
          The results in the electron (left) and muon (right) decay channel are reported.}
	\label{fig:ttbarFitmultijet}
\end{figure}

\subsubsection{Transverse mass distribution fit}
\label{sec:transversemassfit}

As it has already been noticed in the previous section when 
looking at the results in the $t\bar{t}$ control regions,
a certain amount of disagreement between simulation and 
real data is to be accepted even after the reweighting 
of the simulated events with the data/MC efficiency scale factors
for the $b$-tagging and lepton selections.
In this section, an attempt to improve the description of the data 
is performed by calculating a set of global scale factors for the 
most important background samples in the signal region 
from a fit procedure based on some
simple and reliable assumptions. 
Several fit procedures have been tried in order to test the stability 
of the results, and the most interesting ones will be described
in the next paragraphs.
The fit is performed on the transverse mass distribution in the range 
from $0$ to $200$~GeV. The transverse mass distribution has been found 
to yield the best possible discriminating power between the different 
samples and consequently the most stable fit results.
This sentence is especially true in the case of the QCD background.
In all the following fit procedures, the QCD background is treated as 
a special case and its normalization is left completely floating
and not correlated with the remaining backgrounds.
This assumption is justified by the fact that the estimation of QCD is 
completely relying on data and the stability of the fit is granted 
in any case by the very low shape affinity between the 
transverse mass distribution of QCD events and the other samples.

\subsubsection{Uncorrelated top-antitop and W + bb fit}
As a starting point, an uncorrelated fit of the $W + bb$ signal and of the 
$t\bar{t}$ samples has been performed by setting their normalizations 
free to float without any constraint during the fit, along with the 
QCD normalization.
All other backgrounds have been kept out of the fitting procedure 
and their contribution estimated from the simulation prediction.
A scale factor has been derived both for the $W + bb$ contribution:
$$c_{W+bb}(W \rightarrow e \nu + bb) = 1.770 \pm 0.211 \mathrm{,}$$
$$c_{W+bb}(W \rightarrow \mu \nu + bb) = 1.575 \pm 0.163 \mathrm{,}$$
and for the $t\bar{t}$ contributions:
$$c_{t\bar{t}}(W \rightarrow e \nu + bb) = 1.107 \pm 0.064 \mathrm{,}$$
$$c_{t\bar{t}}(W \rightarrow \mu \nu + bb) = 1.136 \pm 0.054 \mathrm{.}$$
These results are graphically represented in figure~\ref{fig:unconstrainedfit}.
\begin{figure}[htb]
	\begin{center}
		\leavevmode
		\includegraphics[width=0.95\textwidth]{figs/trieste_plots/fit1}
	\end{center}
	\caption{Results of fit to the $t\bar{t}$ and $W+bb$ components in the electron (left) and 
          muon (right) channel.}
	\label{fig:unconstrainedfit}
\end{figure}
The $c_{W+bb}$ does not have any effect on the cross section measurement, 
as it is the case instead for the background samples scale factors which 
modify the background contributions to be subtracted from the data;
the formula used to perform the background subtraction and cross section 
calculation will be described in the next chapter. 
As a consequence, $c_{W+bb}$ has to be regarded solely as an indication of the 
level of agreement to be expected between the theoretical and measured 
cross section.
In the case of the the $c_{t\bar{t}}$, the results coming from the 
$t\bar{t}$ control regions have been deliberately ignored: 
the results of the fit are in good agreement with the results coming from 
the $t\bar{t}$ multi-jet control region. Such a behaviour may be 
explained by the fact that the phase space in which the $t\bar{t}$ multi-jet 
events are defined is most near to the signal definition.
Consequently, in the following results, 
the $c_{t\bar{t}}^{\mathrm{multi-jet}}$ will be used when 
constraining the $t\bar{t}$ contribution to the results in the control 
region.
This assumption may be biased by the fact that the other backgrounds are 
not varied during the fitting procedure. As it can be explained with the following results, 
the amount of such a bias is small due the fact that $t\bar{t}$ is by 
far the dominant background and the fit results are stable when 
introducing additional degrees of freedom.

\subsubsection{Top-antitop constrained fit}
In this second example the same kind of fit has been performed 
after constraining the $t\bar{t}$ contribution to the results 
obtained in the multi-jet control region:
$$c_{t\bar{t}}(W \rightarrow e \nu + bb) = 1.141 \pm 0.016 \mathrm{,}$$
$$c_{t\bar{t}}(W \rightarrow \mu \nu + bb) = 1.118 \pm 0.013 \mathrm{.}$$
Not surprisingly, the output of the fitting procedure for the 
$c_{W+bb}$ scale factor is consistent within the statistical uncertainties
with the results from the previous fit: 
$$c_{W+bb}(W \rightarrow e \nu + bb) = 1.691 \pm 0.153 \mathrm{,}$$
$$c_{W+bb}(W \rightarrow \mu \nu + bb) = 1.614 \pm 0.118 \mathrm{.}$$
The the results are graphically represented in figure~\ref{fig:intermediatefit1}.
\begin{figure}[htb]
	\begin{center}
		\leavevmode
		\includegraphics[width=0.95\textwidth]{figs/trieste_plots/fit2}
	\end{center}
	\caption{Results of fit to the $W + bb$ component with $t\bar{t}$ constrained to the 
          results in the multi-jet control region. Electron (left) and 
          muon (right) decay channel.}
	\label{fig:intermediatefit1}
\end{figure}

\subsubsection{Top-antitop and W + bb correlated fit}
Subsequently, a fit has been performed to the 
transverse mass distribution for the $t\bar{t}$ and $W + bb$ 
samples, with the $t\bar{t}$ normalization constrained to the 
measured cross section reported in table~\ref{tab:mcxsecs}.
The $3.5\%$ uncertainty on the $t\bar{t}$ cross section has been
adopted to constrain the $t\bar{t}$ sample.
An additional degree of freedom has been added to the fit by 
including a common $c_{\mathrm{scale}}$ correlating the $t\bar{t}$ and 
$c_{W+bb}$ samples and has been constrained to the 
values of $c_{t\bar{t}}^{\mathrm{multi-jet}}$ obtained in the 
$t\bar{t}$ multi-jet control region:
$$c_{\mathrm{scale}}(W \rightarrow e \nu + bb) = 1.141 \pm 0.016 \mathrm{,}$$
$$c_{\mathrm{scale}}(W \rightarrow \mu \nu + bb) = 1.118 \pm 0.013 \mathrm{,}$$
As expected, the result of the fit for the $c_{t\bar{t}}$
scale factors is consistent with unity:
$$c_{t\bar{t}}(W \rightarrow e \nu + bb) = 0.991 \pm 0.030 \mathrm{,}$$
$$c_{t\bar{t}}(W \rightarrow \mu \nu + bb) = 1.006 \pm 0.029 \mathrm{.}$$
The signal $W+bb$ scale factor has been modified accordingly:
$$c_{W+bb}(W \rightarrow e \nu + bb) = 1.501 \pm 0.154 \mathrm{,}$$
$$c_{W+bb}(W \rightarrow \mu \nu + bb) = 1.432 \pm 0.124 \mathrm{.}$$
The results are graphically represented in figure~\ref{fig:intermediatefit2}.
\begin{figure}[htb]
	\begin{center}
		\leavevmode
		\includegraphics[width=0.95\textwidth]{figs/trieste_plots/fit3}
	\end{center}
	\caption{Results of fit to the $t\bar{t}$ and $W+bb$ components correlated by 
          a common scale factor. The $t\bar{t}$ is constrained to the results 
          in the multi-jet control region. Electron (left) and 
          muon (right) channel.}
	\label{fig:intermediatefit2}
\end{figure}
The results of this fit are, in practice, equivalent to the previous results 
with the only difference in the fact that the modification of the $t\bar{t}$
contribution is delegated to the $c_{\mathrm{scale}}$ scale parameter that 
is in common with the signal sample; as a consequence, the fit converges to a 
smaller value of $c_{W+bb}$, which however does not affect the final cross section 
measurement.
Indeed, this step aims at demonstrating a correlated general underestimation of data 
in simulation. Such a correlation is compatible with a systematic effect in the calibration 
and selection of the objects in the selected final state, common to all simulated
samples; the jet energy corrections and the $b$-tagging efficiency scale factors
may provide an explanation for this effect, since they have a big associated 
systematic uncertainty which will be described in the next chapter.

\subsubsection{Baseline results}
In this paragraph the fitting procedure adopted to estimate the 
cross section is described.
In the same fashion as for the previous configuration, 
the $t\bar{t}$ normalization is constrained to its experimental 
value and uncertainty.
The $c_{\mathrm{scale}}$ global scale is constrained to the 
results in the $t\bar{t}$ multi-jet control region:
$$c_{\mathrm{scale}}(W \rightarrow e \nu + bb) = 1.141 \pm 0.016 \mathrm{,}$$
$$c_{\mathrm{scale}}(W \rightarrow \mu \nu + bb) = 1.118 \pm 0.013 \mathrm{,}$$
and it is now correlating the 
variation of all the simulated samples with the only exception 
of the QCD sample: top-antitop, Single~top, the light and $c$ fractions 
of the $W+$jets sample, the Drell-Yan sample and the di-boson samples.
This modification has the effect of further reducing the 
$W+bb$ signal strength:
$$c_{W+bb}(W \rightarrow e \nu + bb) = 1.419 \pm 0.173 \mathrm{,}$$
$$c_{W+bb}(W \rightarrow \mu \nu + bb) = 1.351 \pm 0.135 \mathrm{,}$$
The results are graphically represented in figure~\ref{fig:baselinefit}.
\begin{figure}[htb]
	\begin{center}
		\leavevmode
		\includegraphics[width=0.8\textwidth]{figs/trieste_plots/electrons/w_mt_wenu_bb_wide_doFit}
		\includegraphics[width=0.8\textwidth]{figs/trieste_plots/muons/w_mt_wmnu_bb_wide_doFit}
	\end{center}
	\caption{Results of fit to the $t\bar{t}$, $W+bb$, $W+$light, Single~$t$ and 
          di-boson samples, correlated by 
          a common scale factor. The $t\bar{t}$ is constrained to the results 
          in the multi-jet control region. Electron (top) and 
          muon (bottom) channel.}
	\label{fig:baselinefit}
\end{figure}
The effects of the normalization scale factors obtained on the fit 
on the number of selected events is reported in table~\ref{tab:yieldsPostfit}, 
to be compared with the pre-fit event yields reported in 
table~\ref{tab:yieldsPrefit}.
\begin{table}[htb]
\begin{center}
\begin{tabular}{|c|c|c|}
\hline
\textbf{Sample} & $W \rightarrow e \nu + bb$ & $W \rightarrow \mu \nu + bb$ \\ 
\hline
DATA        & 5073   & 5322 \\
\hline             
$W + bb$    & 1182.2 & 1316.7 \\
$W + c$     & 78.3   & 82.7 \\
$W + light$ & 153.5  & 154.4 \\
$W + (\rightarrow \tau \nu) + bb$  & 17.5 & 25.4 \\
$t\bar{t}$  & 2315.1 & 2566.4 \\
Single $t$  & 667.5  & 783.8 \\
$VV$        & 106.0  & 126.8 \\
$Z + jets$  & 96.1   & 144.7 \\
\hline
\end{tabular}
\end{center}
\caption{Selected event yields after the baseline fitting procedure described in 
  section~\ref{sec:transversemassfit}.} 
\label{tab:yieldsPostfit}
\end{table}
%-------------------



%-------------------
\subsection{Detector level distributions}
\label{sec:detectorlevel}

The interesting kinematical observables of the $W + bb$ system are presented here
for the selection described in section~\ref{sec:wbbselection} performed on the 
data and Monte Carlo samples listed in section~\ref{sec:dataset}.
The signal component has been selected in the sample of $W + bb$ generated with 
\textsc{MadGraph} in the $4$-flavour number scheme.
The simulated events have been reweighted with the procedures 
described in section~\ref{sec:scalefactors},
in order to achieve the best possible 
agreement with real data.
The signal $W + bb$ contribution, along with the contribution of top-antitop, QCD and 
Single~top background processes, has been estimated with the fitting procedure 
described in the previous section (sec.~\ref{sec:transversemassfit}).
The distributions are shown at detector level, 
meaning that no effort has been spent in
deconvolving the detector response function from the measured distributions.
Results are shown both for the $W \rightarrow e \nu + bb$ and for the 
$W \rightarrow \mu \nu + bb$.
The following list includes all the relevant kinematical distribution of 
the $W + bb$ system presented here:
\begin{itemize}
  \item the transverse momentum $p_{T}$ of the first and of the second $b$-jet
    in the event, in order of decreasing $p_{T}$, are shown in 
    figures~\ref{fig:1stjetpt} and~\ref{fig:2ndjetpt};
  \item the pseudorapidity $\eta$ of the first and of the second $b$-jet, 
    in figures~\ref{fig:1stjeteta} and~\ref{fig:2ndjeteta};
  \item the $p_{T}$ and $\eta$ of the di-jet system, defined as the 
     two $b$-jets four-momentum vector sum in figures~\ref{fig:dijetpt} and~\ref{fig:dijeteta};
  \item the transverse momentum $p_{T}$ of the $W$ boson, defined as the 
    vector sum of the $\vec{E}_{T}^{miss}$ and the $p_{T}$ of the lepton,
    in figure~\ref{fig:Wpt};
  \item the angular distance $\Delta R$ in the $(\phi,\eta)$~space between the two 
    $b$-jets, in figure~\ref{fig:bjetDeltaR};
  \item the polar angle $\Delta \phi$ between the two 
    $b$-jets, in figure~\ref{fig:bjetDeltaPhi};
  \item the jet $H_{T}$ observable, defined as the scalar sum of the 
    transverse momentum of all the jets in the event ($H_{T} = \sum_{i = jets} p_{T}(i)$),
    in figure~\ref{fig:jetHt};
\end{itemize}

\begin{figure}[htb]
	\begin{center}
		\leavevmode
		\includegraphics[width=0.8\textwidth]{figs/trieste_plots/electrons/w_first_jet_pt_bb}
		\includegraphics[width=0.8\textwidth]{figs/trieste_plots/muons/w_first_jet_pt_bb}
	\end{center}
	\caption{Spectrum of the most energetic $b$-jet $p_{T}$. 
          The electron decay channel is on the top plot, 
          the muon decay channel on the bottom.}
	\label{fig:1stjetpt}
\end{figure}

\begin{figure}[htb]
	\begin{center}
		\leavevmode
		\includegraphics[width=0.8\textwidth]{figs/trieste_plots/electrons/w_second_jet_pt_bb}
		\includegraphics[width=0.8\textwidth]{figs/trieste_plots/muons/w_second_jet_pt_bb}
	\end{center}
	\caption{Spectrum of the second $b$-jet $p_{T}$. 
          The electron decay channel is on the top plot, 
          the muon decay channel on the bottom.}
	\label{fig:2ndjetpt}
\end{figure}

\begin{figure}[htb]
	\begin{center}
		\leavevmode
		\includegraphics[width=0.8\textwidth]{figs/trieste_plots/electrons/w_first_jet_eta_bb}
		\includegraphics[width=0.8\textwidth]{figs/trieste_plots/muons/w_first_jet_eta_bb}
	\end{center}
	\caption{Pseudorapidity of the most energetic $b$-jet. 
          The electron decay channel is on the top plot, 
          the muon decay channel on the bottom.}
	\label{fig:1stjeteta}
\end{figure}

\begin{figure}[htb]
	\begin{center}
		\leavevmode
		\includegraphics[width=0.8\textwidth]{figs/trieste_plots/electrons/w_second_jet_eta_bb}
		\includegraphics[width=0.8\textwidth]{figs/trieste_plots/muons/w_second_jet_eta_bb}
	\end{center}
	\caption{Pseudorapidity of the second $b$-jet. 
          The electron decay channel is on the top plot, 
          the muon decay channel on the bottom.}
	\label{fig:2ndjeteta}
\end{figure}

\clearpage

\begin{figure}[htb]
	\begin{center}
		\leavevmode
		\includegraphics[width=0.8\textwidth]{figs/trieste_plots/electrons/w_dijet_pt_bb}
		\includegraphics[width=0.8\textwidth]{figs/trieste_plots/muons/w_dijet_pt_bb}
	\end{center}
	\caption{The $p_{T}$ of the four-vector sum of the two $b$-jets in the event. 
          The electron decay channel is on the top plot, 
          the muon decay channel on the bottom.}
	\label{fig:dijetpt}
\end{figure}

\begin{figure}[htb]
	\begin{center}
		\leavevmode
		\includegraphics[width=0.8\textwidth]{figs/trieste_plots/electrons/w_dijet_eta_bb}
		\includegraphics[width=0.8\textwidth]{figs/trieste_plots/muons/w_dijet_eta_bb}
	\end{center}
	\caption{The pseudorapidity of the four-vector sum of the two jets in the event. 
          The electron decay channel is on the top plot, 
          the muon decay channel on the bottom.}
	\label{fig:dijeteta}
\end{figure}

\begin{figure}[htb]
	\begin{center}
		\leavevmode
		\includegraphics[width=0.8\textwidth]{figs/trieste_plots/electrons/w_pt_W_wenu_bb}
		\includegraphics[width=0.8\textwidth]{figs/trieste_plots/muons/w_pt_W_wmnu_bb}
	\end{center}
	\caption{The $p_{T}$ of the $W$ boson. 
          The electron decay channel is on the top plot, 
          the muon decay channel on the bottom.}
	\label{fig:Wpt}
\end{figure}

\begin{figure}[htb]
	\begin{center}
		\leavevmode
		\includegraphics[width=0.8\textwidth]{figs/trieste_plots/electrons/w_deltaR_wenu_2b}
		\includegraphics[width=0.8\textwidth]{figs/trieste_plots/muons/w_deltaR_wmnu_2b}
	\end{center}
	\caption{The angular distance $\Delta R$ between the two $b$-jets in the event. 
          The electron decay channel is on the top plot, 
          the muon decay channel on the bottom.}
	\label{fig:bjetDeltaR}
\end{figure}

\begin{figure}[htb]
	\begin{center}
		\leavevmode
		\includegraphics[width=0.8\textwidth]{figs/trieste_plots/electrons/w_delta_wenu_2b}
		\includegraphics[width=0.8\textwidth]{figs/trieste_plots/muons/w_delta_wmnu_2b}
	\end{center}
	\caption{The azimuthal distance $\Delta\phi$ between the two $b$-jets in the event. 
          The electron decay channel is on the top plot, 
          the muon decay channel on the bottom.}
	\label{fig:bjetDeltaPhi}
\end{figure}

\begin{figure}[htb]
	\begin{center}
		\leavevmode
		\includegraphics[width=0.8\textwidth]{figs/trieste_plots/electrons/w_Ht_bb}
		\includegraphics[width=0.8\textwidth]{figs/trieste_plots/muons/w_Ht_bb}
	\end{center}
	\caption{The jet $H_{T}$ of the event, defined as the scalar sum of the 
          $p_{T}$ of all the jets in the event. 
          The electron decay channel is on the top plot, 
          the muon decay channel on the bottom.}
	\label{fig:jetHt}
\end{figure}
