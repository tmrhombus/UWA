\section{Introduction}\label{Intro}
The analysis uses a sample of proton-proton collisions 
at $\sqrt{s}=8\TeV$ collected in 2012 with the CMS experiment at the LHC 
and  corresponding to an 
integrated luminosity of $19.8\fbinv$. In order to select b-jets, this analysis uses
the Combined Secondary Vertex algorithm at the tight b-tagging working point. 



The measurement of \cPZ$/\cPgg^*$ (henceforth denoted by ``\cPZ'') or
\Wpm production in association with b quarks in proton-proton
collisions at the Large Hadron Collider (LHC) is relevant for various
experimental searches. In particular, these processes constitute a
background to standard model (SM) Higgs production associated with a
\cPZ~or a \Wpm boson, where the Higgs boson decays into a $b\bar{b}$
pair. The discovery by the ATLAS and Compact Muon Solenoid (CMS)
experiments of a neutral boson with a mass of about
$125\GeV$~\cite{Aad:2012tfa,Chatrchyan:2012ufa} motivates further studies to
establish its nature and determine the coupling of the new boson to b
quarks. Furthermore, different models based on an extension (minimal
or not) of the Higgs sector are being tested against the LHC data
through final states composed of lepton(s) and b-jet(s).  In this
context, a better understanding of the b-hadrons and/or associated
jets dynamics, as well as of the vector boson is required to refine
the background prediction and increase the sensitivity to new physics.

The leading order(LO) diagram describing the $\Wbb$ production is  
shown in Fig.~\ref{fig:WbbDiag} and the $\bbbar$ pair is produced from initial gluon radiation.
This diagram can be compared  
to the associated Higgs
production diagram presented in Fig.~\ref{fig:WHbbDiag}. 

\begin{figure}[!hbtp]
    \center
    \subfloat[\Wmnbb]{\label{fig:WbbDiag}\includegraphics[height=4cm]{figs/diagrams/Wbb.pdf}}
    \subfloat[\WmnH]{\label{fig:WHbbDiag}\includegraphics[height=4cm]{figs/diagrams/WH.pdf}}
    \caption{LO feynman diagrams of production of \Wmnbb and associate production of the Higgs boson in \WmnH. 
\Wmnbb is an irreducible backround in the measurement of $\HBB$ production.}
    %\caption{Leading Order Decay Mode for \Wmnbb}
    \label{fig:diagrams}
  %\end{center}
\end{figure}

%Tree-level calculations allowing for large numbers of extra
%partons in the matrix elements (as initial- and final-state
%radiations) are available. These are provided by
%\MADGRAPH~\cite{Alwall:2007st,Alwall:2011uj},
%{\ALPGEN~\cite{Mangano:2002ea}, and \SHERPA~\cite{Gleisberg:2008ta},
%in both the five- and four-flavour approaches, \ie by considering the
%five or four lightest quark flavours in the proton parton distribution
%function (PDF) sets. Next-to-leading-order (NLO) calculations have
%been performed in both the five-flavour (\MCFM)~\cite{Campbell:2000bg}
%and four-flavour~\cite{FebresCordero:2008ci,Cordero:2009kv}
%approaches. A fully-automated NLO computation matched to a parton
%shower simulation is implemented by the a\MCATNLO event
%generator~\cite{Frederix:2011qg,Frixione:2002ik}. A detailed
%discussion of b-quark production in the different calculation schemes
%is available in Ref.~\cite{Maltoni:2012pa}.

%The production mechanism of $\bbbar$ pairs together with $\PW$ or $\cPZ$ bosons has been the subject of extensive theoretical studies and is included in different simulation programs~\cite{Oleari:2011ey,Frederix:2011qg,Madgraph5}, but is still not thoroughly understood. 

In the past, the production of \cPZ+1 and 2 b
jets~\cite{Chatrchyan:2012vr,Chatrchyan:2014dha,Chatrchyan:2013zja,Aad:2011jn,Aad:2014dvb},
and \Wpm+ b jets~\cite{Chatrchyan:2013uza,Aad:2013vka} have been
studied by the ATLAS and CMS collaborations, at a centre-of-mass
energy $\sqrt{s}$ of 7 TeV using up to 5~fb$^{-1}$ of integrated
luminosity. This analysis  updates the conclusions of the $\sqrt{s}=7\TeV$ analysis of  \Wpm+ b jets, using  8\TeV collected by the CMS detector in 2012, corresponding to integrated luminosity of 19.8\fbinv.
In addition the current analysis also uses both muon and electron $\PW$ decay channels, whereas the former one used only muons.

The event generators used to simulate the signal and
background event samples are used as follows: the production of a vector boson or a pair of top-antitop pair in association with jets
(resp. V+jets, with V=W,Z, and $\ttbar$ + jets) is performed at tree-level using \MADGRAPH~5.1~\cite{Madgraph5} interfaced with
\PYTHIA~6.4.24~\cite{Sjostrand:2006za}.
Since multiple partonic multiplicities are used, the $K_T$-MLM merging scheme is used, assuming a matching scale of 20 and 30 GeV, respectively
for hadronization.
The V+jets samples are generated using the five-flavor scheme, which includes massless b quarks in the initial state.

Single-top-quark event samples are generated at next-to-leading order (NLO) with
\POWHEG~2.0~\cite{Alioli:2008gx,Nason:2004rx,Frixione:2007vw}.
Diboson ($\WW$, $\WZ$, $\ZZ$) samples are
generated with
\PYTHIA~6.4.24.
For LO generators, the default PDF set used is CTEQ6L~\cite{CTEQ66}, while for NLO generators the
showering of partons and hadronization are simulated with \PYTHIA using the Z2
tune~\cite{Field:2010bc}.
For all processes, the detector response is simulated using a detailed
description of the CMS detector based on  \GEANTfour~\cite{GEANT}.
The reconstruction of simulated events is performed with
the same algorithms used for the analyzed data sample.
The simulated event samples include additional minimum-bias interactions per bunch crossing
(pileup).
