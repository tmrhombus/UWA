\section{Background estimation}\label{BkgControl}

After all selection requirements are applyed the major background contributions are:
$\PW$+jets,$\ttbar$, single top, $\cPZ$+jets and QCD. Contributions from these backgrounds are obtained in 
a simultaneus fit, which also provides a final estimate for signal and background yields.
The shapes of the backgrounds are all taken from MC, with the exception of QCD which has been described above. The initial yield is also
taken from MC, normilized to the NNLO predicitons and allowed to vary in the fit within
uncertainites which are described in the next section.
To prove the validity of the MC shapes and normalizations a set of control regions is 
provided for the background contributions.

\subsection{W+jets: light and charm component}

$\PW$+jets is the dominating background before
applying the b-tag requirements.
The MC describes well both the shape
and event yield of the $\PW$+jet events as shown
in Fig.~\ref{fig:wjj_plots}, where the $\pt$ 
of the highest $\pt$ jet is shown along with the $\pt$ and $\eta$ of the
muon and $\MT$ of the reconstructed $\PW$.

\begin{figure}
 \center
 \subfloat[]{\label{fig:initialmuonpt}\includegraphics[width=0.4\textwidth]{figs/plots/2j0b_mu_pt_prefit.png}}
 \subfloat[]{\label{fig:initialmuoneta}\includegraphics[width=0.4\textwidth]{figs/plots/2j0b_mu_eta_prefit.png}}
 \\
 \subfloat[]{\label{fig:initialmt}\includegraphics[width=0.4\textwidth]{figs/plots/2j0b_mt_prefit.png}}
 \subfloat[]{\label{fig:initialj1pt}\includegraphics[width=0.4\textwidth]{figs/plots/2j0b_J1_pt_prefit.png}}
 \caption{
  Selecting for a Tight ID muon with $p_{T}>$30 GeV and exactly two central PFchs-Jets passing Loose ID,
   we recover the distributions shown above. 
  Shown in (\ref{fig:initialmuonpt}) is the transverse momentum of the muon
   and (\ref{fig:initialmuoneta}) is the $\eta$ distribution.
  The transverse Mass of the reconstructed W boson is given in (\ref{fig:initialmt})
   and (\ref{fig:initialj1pt}) shows the transverse momentum of the leading jet.
 } 
    \label{fig:wjj_plots}
\end{figure}

The true $\Wc$ contribution in the signal region is minimal, 
and only possible due to mistaging of a second jet in
the event. However, the contribution of events
with one hard charm and one hard anti-charm originated by gluon 
splitting is not negligable and moreover these events have kinematics closely related 
to that of our signal. 

\subsection{Top backgrounds}
\label{section:topbackgrounds}

To validate the description of the $\ttbar$ contribution two 
control regions are defined and referred to as multilepton 
and multijet $\ttbar$ regions.
The selections for the multijet region are the same as 
those for the $\Wbb$ region except that the veto on the third jet is removed. 
As can be seen in Figure \ref{fig:ttbar_multijet_mt_presel}, this control region is dominated by $\ttbar$.
The selections for the multilepton region differ from those for the $\Wbb$ region in that exactly two well-isolated opposite-flavor leptons are required. 
Figure \ref{fig:ttbar_multilepton_mt_presel} shows this control region,
and because of its purity, this region is used ultimately to constrain the fit.

%The agreement in this final state can be studied by requiring atleast one Isolated Muon 
%and four jets, two of which are \(b\)-tagged and two untagged. This selection and good 
%agreement is shown in Figure \ref{fig:ttbarCR}.
%The two untagged jets originate from the decay of a \(W\)\rightarrow \)c-quark s-quark,
%therefore, a comparison of the reconstruction of the mass of the untagged dijet to the \(W\) mass indicates 
%good jet energy resolution.  As shown in Figure \ref{fig:ttbarJE}, the peak is 
%aligned with the \(W\) mass. % so the jet energy scale is well calibrated
%1619/1768 = 91.57 ttbar / MC  or 1619/1901 = 85.17% ttbar / Data   
\begin{figure}
      \center
      \subfloat[multijet]   {\label{fig:ttbar_multijet_mt_presel}\includegraphics[width=0.4\textwidth]{figs/plots/3j2b_mt_prefit.png}}
      \subfloat[multilepton]{\label{fig:ttbar_multilepton_mt_presel}\includegraphics[width=0.4\textwidth]{figs/plots/2j2b_1m1e_mt_prefit.png}}
      \caption{The two $\ttbar$ control regions are shown here. Multijet, requiring at least 3 jets and multilepton keeping the jet veto and requiring two opposite-flavor leptons.
      }
      \label{fig:ttbar_presel}
\end{figure}


%The single top control region is defined by usual selection requirements, without lepton and jet vetos, but
%in this case only one of the two highest-$\pt$ jets is required to have b-tag and the other one should have
%$\eta > 2.8$. The agreement between data and MC for this control region is demonstrated in the
%Fig.~\ref{fig:sTopMT}. In the fit the single top contribution shape and normalization is taken from the 
%the MC simulation.
%%
%%
%%\begin{figure}
%%      \center
%%      \includegraphics[height=10cm]{fig/backgrounds/MtCal_SingleTop.pdf}
%%      \includegraphics[height=10cm]{fig/backgrounds/highestJetPt_J1Pt_top.pdf}
%%      \caption{The single top control region defined by one \(b\)-tagged jet and one forward jet at \(\eta>2.8\).
%%Both the transverse mass of the selected W in the $t\rightarrow b \mu \nu$ process (left) and the pt of the leading jet (right) show remarkable agreement with the
%%Monte Carlo prediction.}
%%      \label{fig:sTopMT}
%%\end{figure}

\subsection{$\Zll + b\bar{b}$ background}

This backround is validated in a control region where the $\Wbb$
selection requirements are applied, but the lepton veto 
is inverted, requiring two isolated, same-flavor leptons 
and the \MT requirement is dropped.
Shown in Figure \ref{fig:z_peak} is the invariant mass of dimuon pairs
and peak is formed at the \MyZ mass,
demonstrating good lepton energy resolution.

\begin{figure}
      \center
      \includegraphics[width=0.4\textwidth]{figs/plots/2j2b_2m_mass_ll_prefit.png}
      \caption{Mass of the delepton pair in the $\Zll$ control region.
      }
      \label{fig:z_peak}
\end{figure}


%Two control regions are defined for Z/$\gamma$+Jets. One is based on the Z+ 2 Jets
%background where 1 jet is outside of the acceptance; this first control region is defined
%as having atleast 2 muons with a combined invariant mass between 70 and 110 GeV. The second
%control region is based on the $Z\rightarrow\mu\mu$ +1 Jet background where 1 muon is not isolated and
%gets taken as a jet. The uncertainty from these regions is taken to be $\pm 5.3\%$. 
%This control region is also used as a comparsion to the SV mass shape in Appendix~\ref{chapter:SVMassShapes}; 
%kinematic distributions can also be found there.


%\subsection{Other Electroweak Backgrounds}
%
%Negligible, these shapes and yields are taken from MC.
%

