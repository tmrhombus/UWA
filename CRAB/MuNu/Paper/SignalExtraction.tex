\section{Signal Extraction}\label{BkgControl2}
\subsection{Fit Methodology}
The goal of this analysis is to study the process W$+\gBB$ where
each b parton hadronizes and forms a jet; to minimize effects due to colinear 
production of b-hadrons and the occurence of one b-hadron falling outside the
accceptance, it is required that the two highest-$p_{T}$ jets are both
b-Tagged. 
The Combined Secondary Vertex Tight working point is chosen to maximize
both the $\Wbb $ signal significance and the signal to background ratio.
It is especially effective at minimizing $\Wcc$ contamination.

The cross-sections for the primary backgrounds are taken from previous studies at 
8 TeV, the shapes are taken from Monte Carlo, as previously mentioned.
These background normalizations are verified in Section ~\ref{BkgControl}.
Each background physics process is included in the fit as a nuisance parameter where the distrbution of the number of events for the process is considered to have a logarithmic normal shape where the uncertainty as quoted in Table \ref{tab:NuisanceTable} are $\pm1\sigma$ during the final fit.\\


\begin{table}[htb]
\begin{center}
\small
\begin{tabular}{|r|l|l|}
\hline
Source    &Nuisance   &Contribution\\
          &Parameter  &to Final Systematics [\%]\\
\hline
Jet Energy Clustering   & $10\%$      &0.0  \\
b-Tag Efficiency        & $12\%$      &2.9  \\
b-Tag Efficiency: Charm & $24\%$      &9.4  \\
b-Tag Fake Rate         & $15\%$      &5.5  \\
Single Top Scale        & $8\%$       &6.1  \\
TTbar Scale             & $20\%$      &38.3 \\
Single AntiTop Scale    & $10\%$      &4.1  \\
Diboson Scale           & $17\%$      &0.4  \\
W+cc                    & $50\%$      &17.6 \\ 
Z+Jets                  & $10\%$      &1.7  \\
Single Top tW Scale     & $30\%$      &20.7 \\
Top Efficiency          & $6\% $      &41.3 \\
Jet Energy Scale        & 1 $\sigma$  &2.0  \\
Luminosity              & 2.6         &1.5  \\
Muon Energy Scale       & $1\sigma$   &80.4 \\
QCD MC                  & $50\%$      &41.4 \\
W+Light Jets            & $50\%$      &15.9 \\
\hline
\end{tabular}
\caption{Systematic Uncertainties}%Add in MC Uncertainties here??
\label{tab:NuisanceTable}
\end{center}
\end{table}





%\begin{table}[htb]
%\begin{center}
%\small
%\begin{tabular}{|c|c|c|}
%\hline
%Source&Contribution&Contribution\\ 
%&to Final Systematics [pb]&to Final Systematics [\%]\\
%\hline
%b-Tag Efficiency& 0.070&11.5\\
%b-Tag Efficiency TOP/C& 0.016&4.5\\
%Luminosity& 0.021&2.2\\
%Jet Energy Scale&0.025&2.0\\
%Jet Energy Clustering&0.005&0.4\\
%Muon Energy Scale + Efficiency& 0.007&0.6\\
%QCD MC&0.022&1.3\\
%Top MC&0.022&2\\
%Other MC Backgrounds&0.014&1.1\\
%Total& 0.18 pb&13.5\%\\
%\hline
%Luminosity& 0.021&2.2\\
%\hline
%%Fit (Shape uncertainty) & $7\%$ \\
%\hline
%\end{tabular}
%\caption{Systematic Uncertainties}%Add in MC Uncertainties here??
%\label{tab:NuisanceTable}
%\end{center}
%\end{table}

%\begin{table}[htb]
%\begin{center}
%\small
%\begin{tabular}{|c|c|}
%\hline
%Source &Contribution\\ 
%~&to Final Systematics\\
%\hline
%b-Tag Efficiency       & $33\%$\\
%b-Tag Efficiency TOP   & $5\%$\\
%b-Tag Efficiency C     & $4\%$\\
%Luminosity             & $0.4\%$\\
%Jet Energy Scale       &  $27\%$\\
%Jet Energy Clustering  & $<1\%$\\
%Muon Energy Scale      &$<1\%$\\
%$t\bar{t}$ MC          &$1.6\%$\\
%W$+c\bar{c}$ MC        &$4\%$\\
%QCD MC                 &$16\%$\\
%Other MC Backgrounds   & $3.7\%$\\
%\hline
%Total& $100\%$\\
%\hline
%\end{tabular}
%\caption{Contribution of each one of the systematic ncertainties. Note that the }
%%Add in MC Uncertainties here??
%\label{tab:NuisanceTable}
%\end{center}
%\end{table}
%

%When fitting nuisance parameters where it is expected that a varying of the 
%value will be propagated to either the  
%Transverse mass of the W or to $H_{T}$ then a shape uncertainty is included. For
%example, when considering the Jet Clustering Energy Uncertainty the 
%Jet Clustering (JEC) threshold is varied by $\pm10\%$ this is then 
%propagated to the Missing $E_{T}$ by varying the JEC threshold and
%re-summing the total energy in the event and using this re-summation 
%to calculate the Missing $E_{T}$ and the Transverse Mass of the W. 
%The shape variation for JEC uncertainty in the Transverse Mass of the W 
%is shown in figure (\ref{fig:UCEVariation}). 

For this study the final yields are extracted via a binned maximum likelihood fit.
Furthermore, to constrain the most prominent backgrounds and reduce the final
systematic uncertainty a combined fit is performed on 
two regions: the signal, $\Wbb$, region which is designed to maximize the $\Wbb$ signal
significance, and is described in Section (\ref{bjetselection}) and an orthogonal background region
referred to in Secion \ref{section:topbackgrounds} as $\ttbar$ multilepton
which was chosen to accurately estimate the $t\bar{t}$ background.

The final signal selection is summarized in Table (\ref{tab:FitRegions}).

\begin{table}[htb]
\center
\begin{tabular}{r|c|c}
\hline
Variable&Signal Region &$t\bar{t}$ Region\\
  \hline
  \hline
  \# Jets $p_{T}>$ 25 GeV $\eta<2.4$&2&$2$\\
  \# Jets $p_{T}>$ 25 GeV $2.4<\eta<4.5$&0&0\\
  \# Isolated Leptons&1&2\\
  Secondary Vertex& 1 per Jet&1 per Jet\\
\hline
\hline
\end{tabular}
\caption{Selection Requirements on $\Wbb $ Signal and $t\bar{t}$ Background Regions.}
\label{tab:FitRegions}
\end{table}

%The fit for the Signal Region is performed on a two dimensional phase space of the 
%Transverse Mass of the selected W boson versus $H_{T}$. Any background process with 
%a contribution to the total yield of over $0.5\%$ is included as a niusance paramenter 
%in the fit. 

The background region is designed to select $t\bar{t}$ events. This is done
by requiring one isolated muon and one isolated electron;
the two highest $p_{T}$ jets are required to be b-tagged with a CSV discriminant 
at the tight working point.
During the fit the assigned uncertainty to the $t\bar{t}$ background is taken to be completely correlated between the signal and background regions. The 
other MC backgrounds are considered uncorrelated and are only included in the 
$t\bar{t}$ shape fit if their contribution is greater than $0.5\%$.

The fit is done simultaneously in the signal and background regions. An iterative 
minimization is performed using Minuit2 and the Migrad minimization 
algorithm to determine the final scale factor value and error. \cite{MNUserGuide}

\subsection{Fit Results \label{sec:SignalExtraction}}

The fitted yields in the signal region for each one of the processes involved can be found in Table \ref{tab:fitYields}, compared to the
original Monte Carlo prediction. 

XXXXX DIFFERENT VERSIONS OF THE FIT XXXXXXXXX

\begin{table}[htb]
\begin{center}
\begin{tabular}{r|l|l|l}
\bf{W+bb} & \multicolumn{3}{c}{Fit Result: r = 1.309 $\pm$ 0.268}\\
{} & \multicolumn{3}{c}{Fit Bias: 1.0004 $\pm$ 0.243}\\
{}         & PreFit  & PostFit & Ratio \\ \hline
W+bb       & 951.66  & 1261.53 & 1.33\\
W+cc       & 123.78  & 111.39  & 0.90\\
W+udscg    & 68.45   & 54.25   & 0.79\\
TTbar      & 2352.17 & 2626.31 & 1.12\\
Top        & 397.63  & 408.37  & 1.03\\
Tbar       & 227.15  & 234.54  & 1.03\\
T tW       & 88.60   & 86.89   & 0.98\\
Drell-Yan  & 102.17  & 107.15  & 1.05\\
Diboson    & 116.24  & 117.16  & 1.01\\
QCD        & 387.06  & 327.32  & 0.85\\
Total MC   & 4814.92 & 5334.91 & 1.11\\
\hline \hline
Data  & \multicolumn{2}{c|}{5335.0} & 1.000
\end{tabular}
\caption{Yields of MC samples before and after the fitting.}
\label{tab:fitYields}
\end{center}
\end{table}


%We perform two versions of the fit. In Fit A, we supress the light and jet contribution to a minimum 
%by selecting two CSVT bjets. The only remaining backgrounds are top-related, and are found to be well separated from the signal using 
%the transverse momentum distribution of the leading jet. 
%In this case, the total number of observed data events in the signal region are $N_{(S+B)}=1110\pm33$. 
%
%In Fit B, we loosen the btagging criteria (2 CSVM bjets) and then fit to the two-dimensional distribution of the 
%$h_T$ and the sum of the invariant masses of the two jets. In this case, the total number of observed data events in the signal region are $N_{(S+B)}=1980\pm44$.
%
%Only Fit A  will be used in the final unfolding of the cross-section in the next section. Fit B is considered to be a cross-check of the robustness of the analysis
%versus $\Wcc$ contamination.
%
%In both cases the top contribution is constrained in a dedicated signal region (4 jets, 2 btagged, Table~\ref{tab:FitRegions})  
%by a fit to the invariant mass of the two non-btagged jets (W Mass).
%
%The signal strength obtained for the $\Wbb$ signal is in very good agreement in the two fits, and measured to be:
%\begin{itemize}
%\item Fit A: 2 CSV Tight b-Tags (Main analysis) 
%\begin{eqnarray}
%\rho_{\Wbb}=0.90 \pm 0.08 (stat.) ^{-0.13}_{+0.16} (syst.)
%\nonumber
%\end{eqnarray}
%\item Fit B: 2 CSV Medium b-Tags (Cross-Check)
%\begin{eqnarray}
%\rho_{\Wbb}=0.87 \pm 0.07 (stat.) ^{-0.13}_{+0.14} (syst.) % This is what comes out of the tool 
%\nonumber
%\end{eqnarray}
%\end{itemize}
%
%\begin{table}[htb]
%\begin{tabular}{|c|c|c| c|c|c| c|c|c|c|}
%\hline
%\multicolumn{9}{|c|}{Fit A: 2 CSVT b-jets } \\
%\hline
%Process                &\Wbb         &W+light         & \Wc($\bar{c}$) &Z+jets          &$t\bar{t}$    &Single Top     & VV             &QCD \\
%
%\hline                 %& 301.7      & 1.              &20             & 29.4          &542.2          &142.0     &16.9           &43.4   \\
%MC prediction          & 332.3       & 1.5             & 21.0          & 30.9          &595.5          &160.3     &18.9           &33.1 \\
%
%\hline
%Fitted                 &300          &1               & 20.3            &32.2          &647          &169.6          &17.1           &33. \\
%Yields                 &$-56/+63 $       &1\pm1      & $\pm 4.1$     & $\pm 2.9$     & $\pm 51.8$  & $\pm 12.8$     & $\pm 3.4$    & $\pm 16$\\
%
%\hline\hline
%\multicolumn{9}{|c|}{Fit B: 2 CSVM b-jets } \\
%\hline
%Process                &\Wbb         &W+light         & \Wc($\bar{c}$) &Z+jets          &$t\bar{t}$    &Single Top     & VV             &QCD \\
%\hline                %& 526      & 30.1       &228.8       & 66.2              &983.8        &307        &31.2                &83.4   \\
%MC prediction          & 557.5       & 18.9           &169.2           & 54.8           &893.8         &273.9           &33.1           &18.9\\          
%\hline
%Fitted                 &479          &19.0             & 266            &68            &928.6          &282.          &31           &20. \\
%Yields                 &$-71/+67 $       &$\pm1.9$       & $\pm 19$     & $\pm 7$     & $\pm 46.4$       & $\pm 27.7$     & $\pm 13.$    & $\pm 5$\\
%\hline
%\end{tabular}
%\caption{Comparison of the Monte Carlo expected contributions for each one of the processes involved in the measurement and the results of the likelihood minimization performed.
%These yields correspond to the signal region (see Table~\ref{tab:FitRegions}).}
%\label{tab:fitYields}
%\end{table}

Figure~\ref{fig:postfits} illustrates the results of the fit in both the Signal and TTbar background regions. 
The observed data distribution is found to be in good agreement with the prediction by the \MADGRAPH+\PYTHIA Monte Carlo. 

\begin{figure}
\center
\subfigure[]{\label{fig:sig_mt_post}\includegraphics[width=0.4\textwidth]{figs/plots/2j2b_mt_fitted.png}}
\subfigure[]{\label{fig:tt_mt_post}\includegraphics[width=0.4\textwidth]{figs/plots/2j2b_1m1e_mt_fitted.png}}
\caption{Results of the fit in the Signal (\ref{fig:sig_mt_post}) and TTbar multilepton (\ref{fig:tt_mt_post}) regions. Shown here are the transverse mass distributions.}
\label{fig:postfits}
\end{figure}


%\begin{figure}[!hbtp]
%    \center
%    \includegraphics[width=0.6\textwidth]{fig/initial/highestJetPt.pdf}
%    \caption{Main analysis. Distribution for the highest jet $P_{T}$ in a tight selection (CVST b-jets). Monte Carlo predictions are 
%    normalized to the results of the fit. }
%    %\caption{Leading Order Decay Mode for \Wmnbb}
%    \label{fig:2tightregion}
%  %\end{center}
%\end{figure}
%
%\begin{figure}[htbp]
%    \center
%    \subfigure[]{\label{fig:finalJ1SVMass}\includegraphics[width=0.45\textwidth]{fig/final/J1SVMass.pdf}}
%    \subfigure[]{\label{fig:finalJ2SVMass}\includegraphics[width=0.45\textwidth]{fig/final/J2SVMass.pdf}} 
%    \caption{Fit B cross-check analysis. Secondary Vertex Mass distributions for the leading (right) and subleading (left) jets with two medium b-tags. The Monte Carlo predictions are 
%normalized to the results of the fit. }
%      \label{fig:postfitdistributionsSV}
%\end{figure}
%
%Figure~\ref{fig:postfitdistributions} shows different aspects of the W and dijet system in the signal region, also normalized to
%the results of our likelihood minimization. The agreement observed with the transverse mass ($M_T$), missing transverse energy ($E_T^{miss}$), W transverse momentum 
%(W~$p_T$)
%and transverse hadronic energy of the system ($H_T$) is shown to agree with Monte Carlo to within 1 sigma. % "Very good" is very unscientific 
%
%\begin{figure}[htbp]
%    \center
%
%    \subfigure[]{\label{fig:finalMtCal}\includegraphics[width=0.45\textwidth]{fig/final/tight/MtCal.pdf}} 
%    \subfigure[]{\label{fig:finalJ1J2DR}\includegraphics[width=0.45\textwidth]{fig/final/tight/J1J2_DeltaR.pdf}} \\  % This is broken!!
%%    \subfigure[]{\label{fig:finalJ1J2DR}\includegraphics[width=0.45\textwidth]{fig/final/tight/deltaR_2CSVT.png}} \\
%    \subfigure[]{\label{fig:finalWPt}\includegraphics[width=0.45\textwidth]{fig/final/tight/WPt.pdf}}
%    \subfigure[]{\label{fig:finalmJJ}\includegraphics[width=0.45\textwidth]{fig/final/tight/mJJ.pdf}}
%% there is some problem with these plots:
%%    \subfigure[]{\label{fig:finalMjj}\includegraphics[width=0.25\textwidth]{fig/final/mJJ.pdf}}
%%    \subfigure[]{\label{fig:finalPtJJ}\includegraphics[width=0.25\textwidth]{fig/final/ptJJ.pdf}}
%    \caption{Distributions of kinematic variables of the W+dijet system in the signal region with two tight b-tags. Monte Carlo predictions are 
%normalized to the results of the fit.}
%      \label{fig:postfitdistributions}
%\end{figure}
%
%More kinematical distributions describing the $\Wbb$ system can be found in Appendix~\ref{chapter:InitialDist}.
%
%\begin{figure}[!hbtp]
%    \center
%    \subfloat[High $p_{T}$ Jet Secondary Vertex Mass]{\label{fig:J1CSVbtag}\includegraphics[height=7cm]{fig/final/J1SVMass.pdf}}
%    \subfloat[Second $p_{T}$ Jet Secondary Vertex Mass]{\label{fig:J2CSVbtag}\includegraphics[height=7cm]{fig/final/J2SVMass.pdf}}
%    \caption{Highest $p_{T}$ and second highest $p_{T}$ Secondary Vertex masses.}
%    %\caption{Leading Order Decay Mode for \Wmnbb}
%    \label{fig:SVMassFinal}
%  %\end{center}
%\end{figure}


%\begin{figure}[!hbtp]
%    \center
%    \subfloat[Transverse Mass of W]{\label{fig:Mt}\includegraphics[height=7cm]{fig/final/MtCal.pdf}}
%    \subfloat[Missing ET]{\label{fig:MET}\includegraphics[height=7cm]{fig/final/met.pdf}}
%    \caption{Transverse Mass and Missing ET.}
%    %\caption{Leading Order Decay Mode for \Wmnbb}
%    \label{fig:MTfinal}
  %\end{center}
%\end{figure}


%\subsection{Additional tests to the stability of the fit}
%
%\begin{table}[htb]
%\begin{tabular}{|c|c|c|c|c|}
%\hline
%              & $M_{jj}$ (W Mass)        &  No  $t\bar{t}$          &  $t\bar{t}$ scaled to dilepton            & Dilepton   \\
%              & Control Region           &  Control Region          &  ($M_{jj}$ only for JES)                  & Control Region                    \\ \hline
%$\rho(\Wbb)$  & $0.90_{-0.16}^{+0.18}$   &  $0.88_{-0.18}^{+0.22}$  &  $0.90_{-0.16}^{+0.19}$                   &  $0.94_{-0.18}^{+0.19}$   \\ \hline
%ttbar expected& 595.5                    & 595.5                    & 535.9                                     &  595.5  \\           
%ttbar fitted  & $647.3\pm8\%$            & $655 \pm8\%$             &  $639.2 \pm7\%$                           &  $621\pm9.5\%$ \\ \hline 
%JES           & $-0.75\pm 0.28 \sigma$   & $-0.5\pm 1.05 \sigma$     &  $-0.75\pm 0.30 \sigma$                   &  $-1.0\pm 0.65 \sigma$  \\ \hline          
%\end{tabular}
%\caption{ Cross-Check fits to test the stability of the ttbar background.
%} 
%\label{tab:ttbarChecks}
%\end{table}
%
