\section{Cross Section Measurement}
\label{sec:xsections}

The cross section has been measured with the following simple formula:
$$ \sigma (W \rightarrow l \nu + bb) = \frac{N_{\mathrm{total}} - N_{\mathrm{background}}}{\epsilon \mathcal{L}} \mathrm{,}$$
where $N_{\mathrm{total}} - N_{\mathrm{background}}$ is the difference between 
the total number of events selected in data with the set of selections described in 
section~\ref{sec:wbbselection} and the number of background events estimated with 
the technique described in section~\ref{sec:transversemassfit}.
The efficiency for selecting events, $\epsilon$,
has been estimated in simulation (see section~\ref{sec:genlevelEfficiency} for details).
The first section of the present chapter 
deals with the estimate of all the experimental 
uncertainties that may lead to a systematic effect in the measured cross 
section. In the second section, the measured inclusive 
cross sections will be shown and compared to the most accurate 
theoretical prediction. Here, the technique adopted in order to 
combine the results in the two separate leptonic decay channels will 
be described as well.
Eventually, in the third section, an interpretation of the measurement 
will be given and future perspective for this analysis evaluated.
%-------------------

\subsection{Systematic and statistical uncertainties}
\label{sec:systunc}

The systematic sources of uncertainty are essentially divided into
five categories:
\begin{enumerate}
   \item jet energy corrections and resolution;
   \item lepton selection data/MC scale factors;
   \item $b$-tagging scale factors;
   \item background subtraction; 
   \item integrated luminosity measurement and pile-up reweighting procedure.
\end{enumerate}

\subsubsection{Jet energy corrections and resolution}
Jet energy correction uncertainties affect the jet energy
measurement and are associated to the L1, L2, L3 
and residual corrections. %described in section~\ref{sec:jetReco}.
The overall effect is propagated by adding in
quadrature the individual contributions to the uncertainty, as evaluated
in dedicated studies~\cite{jetmet010}. The systematic effect due
to the application of each type of JEC is estimated by comparing 
the final yield of selected events
obtained using the standard JEC and its variations
by plus or minus one standard deviation.
By performing this measurement, the JEC uncertainty 
has been found to contribute for $15\%$ and $18\%$, respectively 
in the electron and muon channel, of the total 
cross section.  \\

In a very similar fashion, the systematic uncertainty related to 
the jet energy resolution scaling factors %described in section~\ref{sec:JER}
 is estimated by comparing
the final number of selected events 
using the standard JER correction and its variations.
The JEC uncertainty has been estimated to be of $5.5\%$ and $3.5\%$
in the electron and muon channel respectively.

\subsubsection{Lepton selection scale factors}

The contribution to the systematic uncertainty 
from the lepton selection scale factor measured with the 
\textit{Tag\&Probe} technique described in Appendix~\ref{sec:tagandprobe}
is estimated with the same technique 
used for the JEC and JER systematics:
the final number of events is estimated
with two different sets of scale factors obtained by adding and subtracting
their associated experimental uncertainty.
The uncertainty on the lepton efficiency scale factor has 
been found to contribute $1.5\%$ (electron channel) 
and $2\%$ (muon channel) to the cross section measurement.

\subsubsection{b-tagging scale factors}

The uncertainty on the data driven $b$-tagging scale factors measurement 
comes mainly from pileup effects, from the the modelling of the $b$ fragmentation 
function and of the gluon splitting phenomenon, and from 
the effects of the kinematical cuts applied during the selection of
$\mu +$~jets events~\cite{btagPOG2}.
The $b$-tagging efficiency scale factors
contribute for $6\%$ of the total cross section to the systematic 
uncertainty in both the leptonic decay channels.

\subsubsection{Background subtraction}

The systematic uncertainty on the background subtraction procedure 
descends from the theoretical uncertainty on the cross sections used to rescale
the simulated background samples and from the uncertainty of the fit 
procedure for the estimate of the background scale parameters described in 
section~\ref{sec:transversemassfit}. 
All these uncertainty contribute for $12\%$ of the total cross section both 
in the electron and muon decay channels.
%This contribution is evaluated by
%performing a comparison between the theoretical errors associated to 
%each of the MC cross sections and to their most recent experimental errors measured by 
%CMS \cite{Czakon:2013goa}\cite{wwwzref}\cite{zzref}. 
%All the used values are summarized in table~\ref{MCxsUnc}. 
%In order to quote the global systematic error, the biggest fluctuation between 
%the experimental and theoretical values is taken as the final uncertainty.

\subsubsection{Integrated luminosity and pile-up reweighting}

The measurement of the integrated luminosity~\cite{luminosity} 
of the data recorded by CMS in 2012
has an associated uncertainty of about $2.6\%$ that affects the global
uncertainty and has thus been taken into account.
The systematic uncertainty coming from the pile-up reweighting
procedure is the propagation of the uncertainty on the total inelastic 
cross section as estimated in simulation and on the measured luminosity 
of the data sample.
Its contribution has been estimated by 
modifying the minimum bias cross section 
by $\pm 5 \%$ and propagating the effects through the 
pileup reweighting procedure and the final results. 
The corresponding uncertainty has been found to be $4\%$ and $5\%$,
respectively, in the electron and muon channels.

\subsubsection{Statistical uncertainties}

The following sources of statistical uncertainty have been taken into account: 
\begin{itemize}
  \item the statistical uncertainty associated to the selected data sample,
  \item the statistical uncertainty related to the simulated background samples 
    used to perform the background subtraction;
  \item the statistic uncertainty on the efficiency measurement 
    (see section~\refal{sec:genlevelEfficiency} for details);
  \item the statistical uncertainties on the $c_{t\bar{t}}$, 
    $c_{\mathrm{QCD}}$ and $c_{\mathrm{scale}}$ scale factors evaluated with 
    the background subtraction fit procedure described in section~\ref{sec:transversemassfit}.
\end{itemize}
All these uncertainties have been summed in quadrature, yielding a global statistical
uncertainty of about $11\%$ and $10\%$ in
the electron and muon channels, respectively.

\subsubsection{Breakdown of the uncertainties}

All the contributions to the systematic uncertainty described in this section are considered 
to be uncorrelated and summed in quadrature.
The detailed breakdown of the systematic and statistic uncertainties on the measured
cross section is presented in
table~\ref{tab:finalxsectab}.

\begin{table}[htb]
\begin{center}
\begin{tabular}{|r|c|c|}
\hline
\textbf{Source} & $W \rightarrow e \nu + bb$ & $W \rightarrow \mu \nu + bb$ \\ 
\hline
JEC                & $15.0\%$  & $18.0\%$ \\ 
JER                & $5.5\%$   & $3.5\%$ \\ 
Lepton SF          & $1.5\%$   & $2.0\%$ \\ 
$b$-tagging SF     & $6.0\%$   & $6.0\%$ \\ 
Background subtr.  & $12.1\%$  & $12.4\%$ \\ 
Pileup             & $4.0\%$   & $5.0\%$ \\ 
Lumi               & $2.6\%$   & $2.6\%$ \\ 
\hline \hline
Statistic          & $10.9\%$   & $9.5\%$ \\ 
\hline
\end{tabular}
%\end{adjustwidth}
\end{center}
\caption{Breakdown of the sources of systematic and statistic uncertainty.} 
\label{tab:finalxsectab}
\end{table}


\subsection{Inclusive cross section measurement}

In this section the results of the cross section measurement are 
eventually presented.
As described in detail in section~\ref{sec:genlevelEfficiency}, 
the cross section measurement has been performed in the particle level 
fiducial region represented by one lepton with transverse momentum 
$p_{T}>30$~GeV and pseudorapidity $|\eta|<2.1$, exactly two $b$-jets 
with transverse momentum $p_{T}>25$~GeV and pseudorapidity $|\eta|<2.4$.
Leptons from $B$~hadron decays have been rejected, or otherwise 
corrected for final state radiation by summing the four-momenta 
of photons in a a cone of radius $R=0.1$ around the lepton.
Jets have been clustered on top of the full list of generated particles 
except neutrinos; jets with a distance $R<0.5$ from a lepton have been 
rejected. At generator level, $b$-jets are tagged when at least one of the 
clustered particles descends from a $B$~hadron and within a cone of radius $R=0.5$
from the jet axis.

\subsubsection{Theoretical prediction}

The expected theoretical cross section in the phase space described above 
has been calculated with \textsc{mcfm~6.8}~\cite{mcfm}
in the massless $5$-flavour scheme at NLO precision 
with the \textsc{MSTW2008NNLO}~\cite{mstw2008} parton density function 
and the renormalization and factorization scales set to the mass of the $W$.
The result of this calculation is the following:
$$\sigma_{\mathrm{TH}} (W \rightarrow l \nu + bb) = 0.51 \pm 0.06~\mathrm{pb.}$$
The uncertainty associated to this result corresponds to the statistical uncertainty of 
the integration step of the calculation.
Activities are ongoing in order to estimate the uncertainties related to the 
particular choice of PDF and factorization and renormalization scales.

\subsubsection{Double parton scattering}

The contribution from double parton scattering (DPS) has been estimated with the 
formula:
$$\sigma_{DPS} = \frac{\sigma_{W} \times \sigma_{bb}}{\sigma_{eff}} \mathrm{.}$$
The values of $\sigma_{W}$ and $\sigma_{bb}$ have been estimated 
in the same fiducial region of the analysis selection:
$$\sigma_{W} = 4361 \pm 79 ~\mathrm{pb,}$$
$$\sigma_{bb} = 0.18 \pm 0.9 ~\mu \mathrm{b.}$$
The estimate of the cross section for the inclusive production of $W$ bosons, $\sigma_{W}$, 
is the output of a NNLO calculation with \textsc{FEWZ} and has a very small 
associated uncertainty, negligible when compared to the uncertainty associated 
to the estimate of the other cross sections in this formula. 
The inclusive $b$~pair production cross section has a much bigger 
associated uncertainty, due to the fact that it has been estimated on a sample of 
events generated with \textsc{MadGraph} and it is accurate up to LO precision.
Efforts are ongoing in order to improve the uncertainty associated to the 
prediction of $\sigma_{bb}$.
The value of $\sigma_{eff}$ has been measured by CMS~\cite{sigmaeffCMS}:
$$\sigma_{eff} = 20.7 \pm 6.6  ~\mathrm{mb,}$$
resulting in a contribution of the DPS to the $W+bb$ cross section of 
$0.4~\pm~0.02$~pb. The DPS corrected theoretical prediction is thus:
%$0.41~\pm~0.024$~pb. The DPS corrected theoretical prediction is thus:
$$\sigma_{\mathrm{TH}} (W \rightarrow l \nu + bb) = 0.55 \pm 0.06~\mathrm{(stat.)}~ \pm 0.02~\mathrm{(DPS)}~\mathrm{pb.}$$

\subsubsection{Experimental measurement}

The experimental results to which the above theoretical prediction 
can be compared are:
$$\sigma (W \rightarrow e \nu + bb) = 0.654 \pm 0.071\mathrm{(stat.)} \pm 0.141\mathrm{(syst.)}~\mathrm{pb.}$$
$$\sigma (W \rightarrow \mu \nu + bb) = 0.656 \pm 0.062\mathrm{(stat.)} \pm 0.153\mathrm{(syst.)}~\mathrm{pb.}$$
respectively in the electron and muon decay channel of the $W$ boson.
%The final statistical uncertainty associated to the measurements takes into account 
%the statistical uncertainty on the data sample as well as the 
%statistical uncertainties on the simulated background samples. 
The quoted statistic and systematic uncertainties comprise all the effects 
described in the previous section. 
As already mentioned in the previous chapter, these uncertainties
do not take into account the effects related to the particular choice of 
the PDF and renormalization scale in the theoretical prediction used to estimate 
the selection efficiency. These effects may have a non negligible impact on the results and 
have been estimated in the previous measurement of $W+bb$ cross section performed 
in CMS on the sample of proton-proton collisions at $7$~TeV and found to contribute 
for the $10\%$ of the total cross section~\cite{cmsWbb7tev}.

\subsubsection{Combination of the results}

The experimental results for the electron and muon decay channels
of the $W$ boson, coming from fully uncorrelated samples of events, 
have been statistically combined in order to reduce the statistical 
uncertainty on the measured cross section:
$$\sigma (W \rightarrow e \nu + bb) = 0.655 \pm 0.047\mathrm{(stat.)} \pm 0.145\mathrm{(syst.)}~\mathrm{pb.}$$
The systematic uncertainty associated to the combined cross section 
has been estimated by separately combining each one of the contributions
described in the previous section.
The systematic uncertainties are not in principle uncorrelated. 
These correlations have been taken into account during the combination 
of the results.
With good approximation, the systematic uncertainties due to 
jet energy corrections and resolution, to the $b$-tagging scale factors, to the 
pileup reweighting and to the luminosity measurement have been 
considered as fully correlated, since they correspond basically to the 
measurement of the same quantity performed on top of two statistically independent 
samples of events.
The uncertainty due to lepton efficiency scale factors instead has been treated as a 
fully uncorrelated quantity between the electron and muon species.

\subsection{Interpretation and perspectives}

The presented results are in good agreement between the measurements performed
in the electron and muon decay channels of the $W$ boson and, within the 
uncertainties, they agree with the cross section theoretical 
prediction computed with \textsc{mcfm} in the massless 5-flavour scheme.
The uncertainty on the experimental results is driven by systematic detector effects,
the most important being the jet related systematic uncertainties.
The sample of events collected by CMS in 2012 at $\sqrt{s}=8$~TeV has proven to be 
appropriate for a first measurement of differential cross sections as a function of the 
most interesting kinematical observables of the $W+bb$ system. 

%Consequently, the differential measurement will be the natural extension to the present 
%work. Activities are ongoing in order to measure the differential cross 
%sections for all the observables of which the detector level distributions have been 
%presented in the previous chapter: namely, the transverse momentum and the pseudorapidity of 
%the $b$-jets and of the $W$ boson, along with the angular distributions of the 
%$W+bb$ system. These results will be unfolded from detector effects in order 
%to obtain particle level differential cross sections that can be possibly compared with 
%the predictions from the most recent physics generators.
%There is a strong interest from the theorists community, in fact, in testing the 
%level of comprehension of the kinematical properties of the $W+bb$ system.
%Such a measurement 
%is very important for understanding better the $b$~quark content of the proton and for 
%testing the accuracy of perturbative QCD calculations involving final states 
%with massive $b$~quarks.
