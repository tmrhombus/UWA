\section{Combine Fit}\label{sec:CombineFit}
In this Appendix, we present an analysis in the $W\rightarrow \mu\nu$
 channel and perform a simultaneous fit using the \Wbb signal region and a $t\bar{t}$ 
 control region in an independant framework from that presented in the body of the text.
We also examine the differences between using a 4-flavour generator-level quark scheme
 and using a 5-flavour scheme as well as the effect of performing the fit in the $t\bar{t}$ 
 multijet phase space in comparison with the $t\bar{t}$ multilepton phase space.

\subsection{Fit Methodology}
Applying the selections described in the text for the Signal, $t\bar{t}$ multilepton
 and $t\bar{t}$ multijet regions, we recover the distributions illustrated in 
 Figure \ref{fig:wbbttbar_prefit} after applying all scale factors. 
Two \Wbb samples are considered, a 4-flavour and a 5-flavour sample. 
The 5-flavour sample is pullled from the inclusive and exclusive W+Jets samples 
 indicated in table \ref{tab:MCSAMPLES}, requiring a b-parton to be found in a selected jet. 
The shape of the 4-flavour sample is taken from the sample indicated in the same table
 and the normalization is taken from the normalization of the 5-flavour sample.

\begin{figure}[htb]
\center
\subfigure[]{\label{fig:4f_prefit}\includegraphics[width=0.4\textwidth]{figs/plots/Wbb4F_1m1e_UUbb_PreFit.png}}
\subfigure[]{\label{fig:5f_prefit}\includegraphics[width=0.4\textwidth]{figs/plots/Wbb5F_1m1e_UUbb_PreFit.png}} 
\\
\subfigure[]{\label{fig:1m1e_prefit}\includegraphics[width=0.4\textwidth]{figs/plots/Wbb4F_1m1e_TTbr_PreFit.png}}
\subfigure[]{\label{fig:3j_prefit}\includegraphics[width=0.4\textwidth]{figs/plots/Wbb4F_3j_TTbr_PreFit.png}} 
\caption{
 The four input distributions used in combination for the four fits.
 In the signal region, the distribution from the \Wbb sample with the 4-flavour quark scheme
  is \ref{fig:4f_prefit} and that from the 5-flavour scheme is \ref{fig:5f_prefit}.
 The unfit $t\bar{t}$ multilepton region is shown in \ref{fig:1m1e_prefit} 
  and the unfit $t\bar{t}$ multijet region is shown in \ref{fig:3j_prefit}.
 }
\label{fig:wbbttbar_prefit}
\end{figure}

The fit is performed using the Higgs Combine Tool\cite{combine}, and here all
 background physics processes are included
 with normalizations floating with independant nuisance parameters. 
The values of these parameters
 are calculated as the quotient of the cross section and its uncertainty and are
 considered to be uncorrelated in the fit.
Systematic uncertainties in the shapes of distributions are also taken into account
 for uncertainties associated with the $b$-tag scale factors, muon energy scale,
 jet energy scale, and the energy scale of the particles which did not get clustered into jets
 but are used in the calculation of \MET.
These uncertainties are correlated across all samples, as is an overall 
 uncertainty in the luminosity.
The distrbution of the
 number of events for the process is considered to have
 a logarithmic normal shape and the numerical values of
 the uncertainties used in the final fit are quoted in Table \ref{tab:NuisanceTable}.
The fit is performed simultaneously in the signal and background regions. An iterative 
minimization is performed using Minuit2 and the Migrad minimization 
algorithm to determine the final scale factor value and error. \cite{MNUserGuide}

\begin{table}[htb]
\begin{center}
\small
\begin{tabular}{r|l|l|l|l|l}
\hline
{}   & {}  & \multicolumn{4}{c}{Contribution to} \\
Source                &Uncertainty& 4F, multilepton & 4F, multijet & 5F, multilepton & 5F, multijet \\
\hline \hline
b-Tag Scale Factor	 & 1$\sigma$  & 0.6$\sigma$ & 1.1$\sigma$ & $0.6\sigma$ & 1.1$\sigma$ \\
Muon Energy Scale	 & 1$\sigma$  & 0.0$\sigma$ & 0.4$\sigma$ & $0.7\sigma$ & 1.0$\sigma$ \\
Jet Energy Scale	 & 1$\sigma$  & 1.2$\sigma$ & 0.4$\sigma$ & $1.0\sigma$ & 0.0$\sigma$ \\
Unclustered Energy Scale & 1$\sigma$  & 0.0$\sigma$ & 0.0$\sigma$ & $0.0$       & 0.0$\sigma$ \\
Luminosity               & $2.6\%$    & $0.4\%$     & $2.4\%$     & $0.6\%$     & $2.5\%$     \\
QCD Scale		 & $50\%$     & $33\%$      & $33\%$      & $25\%$      & $26\%$      \\
Drell-Yan+Jets Scale	 & $7.9\%$    & $0.1\%$     & $0.0\%$     & $0.5\%$     & $0.5\%$     \\
Diboson Scale            & $8.1\%$    & $0.2\%$     & $0.1\%$     & $0.4\%$     & $0.4\%$     \\
W+cc Scale		 & $8.1\%$    & $0.3\%$     & $0.4\%$     & $0.4\%$     & $0.3\%$     \\ 
W+Light Jets Scale       & $13.2\%$   & $7.1\%$     & $8.7\%$     & $4.4\%$     & $6.2\%$     \\
Single AntiTop Scale     & $5.4\%$    & $0.6\%$     & $0.5\%$     & $0.6\%$     & $0.5\%$     \\
Single Top Scale         & $4.2\%$    & $0.1\%$     & $0.0\%$     & $0.3\%$     & $0.1\%$     \\
TTbar Scale              & $3.5\%$    & $1.7\%$     & $5.2\%$     & $2.1\%$     & $5.6\%$     \\
\hline
\end{tabular}
\caption{Listed in the first column are the ystematic uncertainties present in the fit, 
  including correlated 1 $\sigma$ shape variations for the first four
  listed, and uncorrelated normalization parameters
  for each of the contributing backgrounds.
 The second column indicates the $1\sigma$ error level set in the fit
  and the remaining four columns indicate the extent to which the variable
  was actually moved in each of the four fits. 
 The fits are labeled as: \Wbb quark flavor sample, $t\bar{t}$ phase space. }
\label{tab:NuisanceTable}
\end{center}
\end{table}

\subsection{Fit Results \label{sec:SignalExtraction}}

In total, four independant fits were performed, one for each combination of
 \Wbb quark scheme and $t\bar{t}$ phase space.
Figures~\ref{fig:postfits_Wbb4F_1m1e},\ref{fig:postfits_Wbb4F_3j},
 \ref{fig:postfits_Wbb5F_1m1e},\ref{fig:postfits_Wbb5F_3j} 
 illustrate the results of the
 fits in both the Signal and TTbar background regions. 
The observed data distribution is found to be in
 good agreement with the prediction by the \MADGRAPH+\PYTHIA Monte Carlo. 
The fitted yields in the signal region for each of the processes
 involved can be found in Tables \ref{tab:fitYields_4F_1m1e},
 \ref{tab:fitYields_4F_3j},\ref{tab:fitYields_5F_1m1e},\ref{tab:fitYields_5F_3j}
 and are compared against the original Monte Carlo prediction. 
The output of these fits are signal strengths of $r=1.369\pm0.137$,
 $r=1.190\pm0.148$, $r=1.357\pm0.135$, $r=1.184\pm0.138$ for 
 4F/multilepton, 4F/multijet, 5F/multilepton, and 5F/multijet fits respectively. 
To verify the stability of the fit, the fit was perfomed repeatedly,
 removing a different uncertainty from each iteration. 
The outcomes of these tests can be seen in Figure \ref{fig:stabilityOfFits}
  and it is clear that the fits have converged to a stable point.

\begin{figure}[htb]
\center
\subfigure[]{\label{fig:sig04_1m1e_mt_post}\includegraphics[width=0.4\textwidth]{figs/plots/Wbb4F_1m1e_UUbb_Fitted.png}}
\subfigure[]{\label{fig:tt04_1m1e_mt_post}\includegraphics[width=0.4\textwidth]{figs/plots/Wbb4F_1m1e_TTbr_Fitted.png}}
\caption{Results of the fit in the Signal region using the 4-Flavour quark scheme (\ref{fig:sig04_1m1e_mt_post})
 and in the TTbar multilepton (\ref{fig:tt04_1m1e_mt_post}) region.
 Shown here are the transverse mass distributions.}
\label{fig:postfits_Wbb4F_1m1e}
\end{figure}

\begin{figure}[htb]
\center
\subfigure[]{\label{fig:sig04_3j_mt_post}\includegraphics[width=0.4\textwidth]{figs/plots/Wbb4F_3j_UUbb_Fitted.png}}
\subfigure[]{\label{fig:tt04_3j_mt_post}\includegraphics[width=0.4\textwidth]{figs/plots/Wbb4F_3j_TTbr_Fitted.png}}
\caption{Results of the fit in the Signal region using the 4-Flavour quark scheme (\ref{fig:sig04_3j_mt_post})
 and in the TTbar multijet (\ref{fig:tt04_3j_mt_post}) region.
 Shown here are the transverse mass distributions.}
\label{fig:postfits_Wbb4F_3j}
\end{figure}

\begin{figure}[htb]
\center
\subfigure[]{\label{fig:sig05_1m1e_mt_post}\includegraphics[width=0.4\textwidth]{figs/plots/Wbb5F_1m1e_UUbb_Fitted.png}}
\subfigure[]{\label{fig:tt05_1m1e_mt_post}\includegraphics[width=0.4\textwidth]{figs/plots/Wbb5F_1m1e_TTbr_Fitted.png}}
\caption{Results of the fit in the Signal region using the 5-Flavour quark scheme (\ref{fig:sig05_1m1e_mt_post})
 and in the TTbar multilepton (\ref{fig:tt05_1m1e_mt_post}) region.
 Shown here are the transverse mass distributions.}
\label{fig:postfits_Wbb5F_1m1e}
\end{figure}

\begin{figure}[htb]
\center
\subfigure[]{\label{fig:sig05_3j_mt_post}\includegraphics[width=0.4\textwidth]{figs/plots/Wbb5F_3j_UUbb_Fitted.png}}
\subfigure[]{\label{fig:tt05_3j_mt_post}\includegraphics[width=0.4\textwidth]{figs/plots/Wbb5F_3j_TTbr_Fitted.png}}
\caption{Results of the fit in the Signal region using the 5-Flavour quark scheme (\ref{fig:sig05_3j_mt_post})
 and in the TTbar multijet (\ref{fig:tt05_3j_mt_post}) region.
 Shown here are the transverse mass distributions.}
\label{fig:postfits_Wbb5F_3j}
\end{figure}

\begin{table}[htb]
\begin{center}
\begin{tabular}{r|l|l|l|l|l|l}
\bf{W+bb} & \multicolumn{6}{c}{Fit Result: r = 1.369 $\pm$ 0.137}\\ 
Wbb4F & \multicolumn{6}{c}{Fit Bias: 1.0000 $\pm$ 0.187}\\ 
1m1e & \multicolumn{3}{c}{Full $m_T$ range} & \multicolumn{3}{c}{$m_T>45$}\\ 
{} & PreFit & PostFit & Ratio & PreFit & PostFit & Ratio \\ \hline 
W+bb  &	 1158.95 &	 1655.08 &	 1.43 &	 882.11 &	 1263.61 &	 1.43\\ 
TTbar  &	 2888.52 &	 3249.23 &	 1.12 &	 2257.96 &	 2540.38 &	 1.13\\ 
T  &	 559.02 &	 612.63 &	 1.10 &	 421.51 &	 460.46 &	 1.09\\ 
Tbar  &	 343.69 &	 377.34 &	 1.10 &	 264.59 &	 289.68 &	 1.09\\ 
W+udscg  &	 164.74 &	 184.89 &	 1.12 &	 146.67 &	 165.53 &	 1.13\\ 
W+cc  &	 52.06 &	 55.65 &	 1.07 &	 41.18 &	 44.39 &	 1.08\\ 
Diboson  &	 153.64 &	 160.07 &	 1.04 &	 112.53 &	 116.65 &	 1.04\\ 
Drell-Yan  &	 166.30 &	 173.50 &	 1.04 &	 107.13 &	 109.96 &	 1.03\\ 
QCD  &	 1308.26 &	 1003.50 &	 0.77 &	 360.65 &	 276.64 &	 0.77\\ 
Total MC  &	 6795.18 &	 7471.89 &	 1.10 &	 4594.33 &	 5267.30 &	 1.15\\ 
\hline \hline 
Data & \multicolumn{2}{c|}{7513.0} & 1.006 & \multicolumn{2}{c|}{5313.0} & 1.009 
\end{tabular}
\caption{
 Yields of MC samples before and after the fitting
  in both the full range of transverse mass and restricted 
  to $\MT>45$GeV.
 Shown above are results from the fit made using the 4-flavour \Wbb sample and the
  $t\bar{t}$ multilepton phase space.}
\label{tab:fitYields_4F_1m1e}
\end{center}
\end{table}

\begin{table}[htb]
\begin{center}
\begin{tabular}{r|l|l|l|l|l|l}
\bf{W+bb} & \multicolumn{6}{c}{Fit Result: r = 1.190 $\pm$ 0.148}\\ 
Wbb4F & \multicolumn{6}{c}{Fit Bias: 1.0000 $\pm$ 0.172}\\ 
3j & \multicolumn{3}{c}{Full $m_T$ range} & \multicolumn{3}{c}{$m_T>45$}\\ 
{} & PreFit & PostFit & Ratio & PreFit & PostFit & Ratio \\ \hline 
W+bb  &	 1158.95 &	 1511.65 &	 1.30 &	 882.11 &	 1151.79 &	 1.31\\ 
TTbar  &	 2888.52 &	 3357.26 &	 1.16 &	 2257.96 &	 2630.61 &	 1.17\\ 
T  &	 559.02 &	 624.60 &	 1.12 &	 421.51 &	 470.49 &	 1.12\\ 
Tbar  &	 343.69 &	 384.12 &	 1.12 &	 264.59 &	 295.57 &	 1.12\\ 
W+udscg  &	 164.74 &	 220.63 &	 1.34 &	 146.67 &	 197.00 &	 1.34\\ 
W+cc  &	 52.06 &	 56.31 &	 1.08 &	 41.18 &	 43.69 &	 1.06\\ 
Diboson  &	 153.64 &	 168.80 &	 1.10 &	 112.53 &	 123.55 &	 1.10\\ 
Drell-Yan  &	 166.30 &	 190.25 &	 1.14 &	 107.13 &	 121.97 &	 1.14\\ 
QCD  &	 1308.26 &	 999.23 &	 0.76 &	 360.65 &	 275.46 &	 0.76\\ 
Total MC  &	 6795.18 &	 7512.85 &	 1.11 &	 4594.33 &	 5310.14 &	 1.16\\ 
\hline \hline 
Data & \multicolumn{2}{c|}{7513.0} & 1.000 & \multicolumn{2}{c|}{5313.0} & 1.001 
\end{tabular}
\caption{
 Yields of MC samples before and after the fitting
  in both the full range of transverse mass and restricted 
  to $\MT>45$GeV.
 Shown above are results from the fit made using the 4-flavour \Wbb sample and the
  $t\bar{t}$ multijet phase space.}
\label{tab:fitYields_4F_3j}
\end{center}
\end{table}

\begin{table}[htb]
\begin{center}
\begin{tabular}{r|l|l|l|l|l|l}
\bf{W+bb} & \multicolumn{6}{c}{Fit Result: r = 1.357 $\pm$ 0.135}\\ 
Wbb5F & \multicolumn{6}{c}{Fit Bias: 1.0000 $\pm$ 0.170}\\ 
1m1e & \multicolumn{3}{c}{Full $m_T$ range} & \multicolumn{3}{c}{$m_T>45$}\\ 
{} & PreFit & PostFit & Ratio & PreFit & PostFit & Ratio \\ \hline 
W+bb  &	 1158.95 &	 1630.28 &	 1.41 &	 903.60 &	 1273.09 &	 1.41\\ 
TTbar  &	 2888.52 &	 3229.35 &	 1.12 &	 2257.96 &	 2526.29 &	 1.12\\ 
T  &	 559.02 &	 607.11 &	 1.09 &	 421.51 &	 456.56 &	 1.08\\ 
Tbar  &	 343.69 &	 374.51 &	 1.09 &	 264.59 &	 287.78 &	 1.09\\ 
W+udscg  &	 164.74 &	 190.30 &	 1.16 &	 146.67 &	 170.25 &	 1.16\\ 
W+cc  &	 52.06 &	 48.92 &	 0.94 &	 41.18 &	 37.52 &	 0.91\\ 
Diboson  &	 153.64 &	 159.47 &	 1.04 &	 112.53 &	 116.34 &	 1.03\\ 
Drell-Yan  &	 166.30 &	 173.67 &	 1.04 &	 107.13 &	 110.25 &	 1.03\\ 
QCD  &	 1308.26 &	 1064.19 &	 0.81 &	 360.65 &	 293.37 &	 0.81\\ 
Total MC  &	 6795.18 &	 7477.80 &	 1.10 &	 4615.82 &	 5271.45 &	 1.14\\ 
\hline \hline 
Data & \multicolumn{2}{c|}{7513.0} & 1.005 & \multicolumn{2}{c|}{5313.0} & 1.008 
\end{tabular}
\caption{
 Yields of MC samples before and after the fitting
  in both the full range of transverse mass and restricted 
  to $\MT>45$GeV.
 Shown above are results from the fit made using the 5-flavour \Wbb sample and the
  $t\bar{t}$ multilepton phase space.}
\label{tab:fitYields_5F_1m1e}
\end{center}
\end{table}

\begin{table}[htb]
\begin{center}
\begin{tabular}{r|l|l|l|l|l|l}
\bf{W+bb} & \multicolumn{6}{c}{Fit Result: r = 1.184 $\pm$ 0.138}\\ 
Wbb5F & \multicolumn{6}{c}{Fit Bias: 1.0000 $\pm$ 0.158}\\ 
3j & \multicolumn{3}{c}{Full $m_T$ range} & \multicolumn{3}{c}{$m_T>45$}\\ 
{} & PreFit & PostFit & Ratio & PreFit & PostFit & Ratio \\ \hline 
W+bb  &	 1158.95 &	 1497.47 &	 1.29 &	 903.60 &	 1168.90 &	 1.29\\ 
TTbar  &	 2888.52 &	 3329.10 &	 1.15 &	 2257.96 &	 2610.00 &	 1.16\\ 
T  &	 559.02 &	 618.32 &	 1.11 &	 421.51 &	 466.09 &	 1.11\\ 
Tbar  &	 343.69 &	 380.79 &	 1.11 &	 264.59 &	 293.29 &	 1.11\\ 
W+udscg  &	 164.74 &	 227.92 &	 1.38 &	 146.67 &	 203.32 &	 1.39\\ 
W+cc  &	 52.06 &	 51.33 &	 0.99 &	 41.18 &	 38.49 &	 0.93\\ 
Diboson  &	 153.64 &	 168.35 &	 1.10 &	 112.53 &	 123.38 &	 1.10\\ 
Drell-Yan  &	 166.30 &	 190.84 &	 1.15 &	 107.13 &	 122.66 &	 1.14\\ 
QCD  &	 1308.26 &	 1056.00 &	 0.81 &	 360.65 &	 291.11 &	 0.81\\ 
Total MC  &	 6795.18 &	 7520.13 &	 1.11 &	 4615.82 &	 5317.24 &	 1.15\\ 
\hline \hline 
Data & \multicolumn{2}{c|}{7513.0} & 0.999 & \multicolumn{2}{c|}{5313.0} & 0.999 
\end{tabular}
\caption{
 Yields of MC samples before and after the fitting
  in both the full range of transverse mass and restricted 
  to $\MT>45$GeV.
 Shown above are results from the fit made using the 5-flavour \Wbb sample and the
  $t\bar{t}$ multijet phase space.}
\label{tab:fitYields_5F_3j}
\end{center}
\end{table}

\begin{figure}[htb]
\center
\subfigure[]{\label{fig:sof_4f_1m1e}\includegraphics[width=0.4\textwidth]{figs/plots/Wbb4F_1m1e_FitStability.png}}
\subfigure[]{\label{fig:sof_4f_3j}\includegraphics[width=0.4\textwidth]{figs/plots/Wbb4F_1m1e_FitStability.png}} 
\\
\subfigure[]{\label{fig:sof_5f_1m1e}\includegraphics[width=0.4\textwidth]{figs/plots/Wbb5F_1m1e_FitStability.png}}
\subfigure[]{\label{fig:sof_5f_3j}\includegraphics[width=0.4\textwidth]{figs/plots/Wbb5F_3j_FitStability.png}}
\caption{The stability of the fit is tested by systematically removing
 fitting parameters and the outcome is above.
 In the first column, we have all parameters included
 and in the subsequent ones, the label indicates the removed parameter.
 The sign (s) indicates a correlated shape uncertainty. Shown here are 
 4F/multilepton (\ref{fig:sof_4f_1m1e}), 4F/multijet (\ref{fig:sof_4f_3j}),
 5F/multilepton (\ref{fig:sof_5f_1m1e}), and 5F/multijet (\ref{fig:sof_5f_3j}) fits respectively.
 }
\label{fig:stabilityOfFits}
\end{figure}


