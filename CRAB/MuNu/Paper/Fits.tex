\section{Fits}\label{sec:Fits}

\subsection{Prefit Signal Region}
\label{sec:backgrounds}

% PREFIT
In Figure~\ref{fig:transvMassPrefit} the transverse mass distribution for events
selected with the prescription described in section~\ref{sec:wbbselection} both
in the electron and the muon decay channels of the $W$ boson is shown.
The black dots indicate the events from data. The colored areas indicate
the contribution from simulated signal (light dashed yellow) and
background events.
The simulated events have been reweighted with the set of Monte Carlo scale factors
described earlier and the numerical
 yields from each contribution are reported in table~\ref{tab:yieldsPrefit}.

\begin{figure}[htb]
        \begin{center}
                \leavevmode
                \includegraphics[width=0.8\textwidth]{figs/trieste_plots/electrons/w_mt_wenu_bb_wide}
                \includegraphics[width=0.8\textwidth]{figs/trieste_plots/muons/w_mt_wmnu_bb_wide}
        \end{center}
        \caption{The transverse mass distribution of events in the signal region for the
        electron (top) and muon (bottom) decay channels of the $W$ bosons.}
        \label{fig:transvMassPrefit}
\end{figure}


\begin{table}[htb]
\begin{center}
\begin{tabular}{|c|c|c|}
\hline
\textbf{Sample} & $W \rightarrow e \nu + bb$ & $W \rightarrow \mu \nu + bb$ \\
\hline
DATA        & 5073   & 5322 \\
\hline
$W + bb$    & 729.3  & 872.0 \\
$W + c$     & 68.6   & 74.0 \\
$W + light$ & 134.5  & 138.1 \\
$W(\rightarrow \tau \nu) + bb$  & 15.4 & 22.6 \\
$t\bar{t}$  & 2028.6 & 2295.9 \\
Single $t$  & 584.9  & 701.2 \\
$WW$        & 92.9   & 113.4 \\
%$WZ$        & - & - \\
%$ZZ$        & - & - \\
$Z + jets$  & 84.2   & 129.4 \\
\hline
\end{tabular}
\end{center}
\caption{Selected number of events in the signal region and with event-by-event scale factors applied.}
\label{tab:yieldsPrefit}
\end{table}

% TTBAR FIT

\subsubsection{Top-antitop control region}
\label{sec:ttbar}

The production of $t\bar{t}$ pairs is by far the predominant background
in the selection of $W + bb$ events and hence a study in two separate
control regions has been dedicated to assessing its contribution.
The multilepton phase space, described in Section \ref{section:topbackgrounds}
is particularly efficient with respect to $t\bar{t}$
fully-leptonic decays and the transverse mass distribution
of the selected events is represented in Figure~\ref{fig:ttbarFitemu}.
The $t\bar{t}$ component has been fitted to the data and a global scale
parameter has been extracted from the fit procedure with the following results:
$$c_{t\bar{t}}^{e/\mu}(W \rightarrow e \nu + bb) = 1.044 \pm 0.31 \mathrm{,}$$
$$c_{t\bar{t}}^{e/\mu}(W \rightarrow \mu \nu + bb) = 1.056 \pm 0.031 \mathrm{,}$$
in the \texttt{SingleElectron} and \texttt{SingleMuon} primary datasets,
respectively, where the associated uncertainty corresponds to
the statistical uncertainty of the fit.
The results of the fit are consistent in the two statistically independent
samples and indicate an excess of data events with respect to the simulation.
The disagreement is nevertheless compatible with the
$t\bar{t}$ measured cross section,
to which the pre-fit sample is normalized, within its corresponding uncertainty
of $3.5\%$.

The strong predominance of $t\bar{t}$ events at higher
jet multiplicities is especially evident when looking at the fraction of simulated
samples as a function of the jet multiplicity in Figure~\ref{fig:jetmultiplicity}.
As a consequence of the different selection,
it is fully uncorrelated from the previous $e/\mu$ control sample.
The sample of selected events with this ``multi-jet''
selection is reported in Figure~\ref{fig:ttbarFitmultijet}.
The scaling factors extracted from the fit of the simulated $t\bar{t}$ sample
to the data are the following:
$$c_{t\bar{t}}^{\mathrm{multi-jet}}(W \rightarrow e \nu + bb) = 1.141 \pm 0.016 \mathrm{,}$$
$$c_{t\bar{t}}^{\mathrm{multi-jet}}(W \rightarrow \mu \nu + bb) =1.118 \pm 0.013 \mathrm{,}$$
and the excess of data is stronger in the multi-jet control sample.
Still, experimental uncertainties have to be taken into account beside the
$3.5\%$ $t\bar{t}$ cross section uncertainty and their contribution
will be described in the next chapter.
For example, the systematic uncertainty due to the $b$-tagging data/MC scale factors alone
is of the order of $6\%$ and may partially explain such a disagreement.


\begin{figure}[htb]
        \begin{center}
                \leavevmode
                \includegraphics[width=0.49\textwidth]{figs/trieste_plots/electronsTOP.old/w_mt_wenu_bb_wide_doFit}
                \includegraphics[width=0.49\textwidth]{figs/trieste_plots/muonsTOP.old/w_mt_wmnu_bb_wide_doFit}
        \end{center}
        \caption{Sample of data and Monte Carlo events selected in the $t\bar{t}$ ``$e/\mu$'' control region.
          The results in the electron (left) and muon (right) decay channel are reported.}
        \label{fig:ttbarFitemu}
\end{figure}

\begin{figure}[htb]
        \begin{center}
                \leavevmode
                \includegraphics[width=0.49\textwidth]{figs/trieste_plots/electronsTOP/w_jetmultiplicity}
                \includegraphics[width=0.49\textwidth]{figs/trieste_plots/muonsTOP/w_jetmultiplicity}
        \end{center}
        \caption{The multiplicity of jets in the sample
          of events selected with the $t\bar{t}$ multi-jet selection
          criteria, described in~\ref{sec:ttbar}.}
        \label{fig:jetmultiplicity}
\end{figure}

\begin{figure}[htb]
        \begin{center}
                \leavevmode
                \includegraphics[width=0.49\textwidth]{figs/trieste_plots/electronsTOP/w_mt_wenu_bb_wide_doFit}
                \includegraphics[width=0.49\textwidth]{figs/trieste_plots/muonsTOP/w_mt_wmnu_bb_wide_doFit}
        \end{center}
        \caption{Sample of data and Monte Carlo events selected in the $t\bar{t}$ ``multi-jet'' control region.
          The results in the electron (left) and muon (right) decay channel are reported.}
        \label{fig:ttbarFitmultijet}
\end{figure}

\subsection{Transverse mass distribution fit}
\label{sec:transversemassfit}

As it has already been noticed in the previous section when
looking at the results in the $t\bar{t}$ control regions,
a certain amount of disagreement between simulation and
real data is to be accepted even after the reweighting
of the simulated events with the data/MC efficiency scale factors
for the $b$-tagging and lepton selections.
In this section, an attempt to improve the description of the data
is performed by calculating a set of global scale factors for the
most important background samples in the signal region
from a fit procedure based on some
simple and reliable assumptions.
Several fit procedures have been tried in order to test the stability
of the results, and the most interesting ones will be described
in the next paragraphs.
The fit is performed on the transverse mass distribution in the range
from $0$ to $200$~GeV. The transverse mass distribution has been found
to yield the best possible discriminating power between the different
samples and consequently the most stable fit results.
This is particularly the case for the QCD background;
in all the following fit procedures, the QCD background is treated as
a special case and its normalization is left completely floating
and not correlated with the remaining backgrounds.
This assumption is justified by the fact that the estimation of QCD is
completely relying on data and the stability of the fit is granted
in any case by the very low shape affinity between the
transverse mass distribution of QCD events and the other samples.

\subsection{Uncorrelated top-antitop and W + bb fit}
As a starting point, an uncorrelated fit of the $W + bb$ signal and of the
$t\bar{t}$ samples has been performed by setting their normalizations
free to float without any constraint during the fit, along with the
QCD normalization.
All other backgrounds have been kept out of the fitting procedure
and their contribution estimated from the simulation prediction.
A scale factor has been derived both for the $W + bb$ contribution:
$$c_{W+bb}(W \rightarrow e \nu + bb) = 1.770 \pm 0.211 \mathrm{,}$$
$$c_{W+bb}(W \rightarrow \mu \nu + bb) = 1.575 \pm 0.163 \mathrm{,}$$
and for the $t\bar{t}$ contributions:
$$c_{t\bar{t}}(W \rightarrow e \nu + bb) = 1.107 \pm 0.064 \mathrm{,}$$
$$c_{t\bar{t}}(W \rightarrow \mu \nu + bb) = 1.136 \pm 0.054 \mathrm{.}$$
These results are graphically represented in Figure~\ref{fig:unconstrainedfit}.
\begin{figure}[htb]
        \begin{center}
                \leavevmode
                \includegraphics[width=0.95\textwidth]{figs/trieste_plots/fit1}
        \end{center}
        \caption{Results of fit to the $t\bar{t}$ and $W+bb$ components in the electron (left) and
          muon (right) channel.}
        \label{fig:unconstrainedfit}
\end{figure}
The $c_{W+bb}$ does not have any effect on the cross section measurement,
as it is the case instead for the background samples scale factors which
modify the background contributions to be subtracted from the data;
the formula used to perform the background subtraction and cross section
calculation will be described in the next chapter.
As a consequence, $c_{W+bb}$ has to be regarded solely as an indication of the
level of agreement to be expected between the theoretical and measured
cross section.
In the case of the the $c_{t\bar{t}}$, the results coming from the
$t\bar{t}$ control regions have been deliberately ignored:
the results of the fit are in good agreement with the results coming from
the $t\bar{t}$ multi-jet control region. Such a behaviour may be
explained by the fact that the phase space in which the $t\bar{t}$ multi-jet
events are defined is most near to the signal definition.
Consequently, in the following results,
the $c_{t\bar{t}}^{\mathrm{multi-jet}}$ will be used when
constraining the $t\bar{t}$ contribution to the results in the control
region.
This assumption may be biased by the fact that the other backgrounds are
not varied during the fitting procedure. As it can be explained with the following results,
the amount of such a bias is small due the fact that $t\bar{t}$ is by
far the dominant background and the fit results are stable when
introducing additional degrees of freedom.



\subsection{Top-antitop constrained fit}
In this second example the same kind of fit has been performed
after constraining the $t\bar{t}$ contribution to the results
obtained in the multi-jet control region:
$$c_{t\bar{t}}(W \rightarrow e \nu + bb) = 1.141 \pm 0.016 \mathrm{,}$$
$$c_{t\bar{t}}(W \rightarrow \mu \nu + bb) = 1.118 \pm 0.013 \mathrm{.}$$
Not surprisingly, the output of the fitting procedure for the
$c_{W+bb}$ scale factor is consistent within the statistical uncertainties
with the results from the previous fit:
$$c_{W+bb}(W \rightarrow e \nu + bb) = 1.691 \pm 0.153 \mathrm{,}$$
$$c_{W+bb}(W \rightarrow \mu \nu + bb) = 1.614 \pm 0.118 \mathrm{.}$$
The the results are graphically represented in Figure~\ref{fig:intermediatefit1}.
\begin{figure}[htb]
        \begin{center}
                \leavevmode
                \includegraphics[width=0.95\textwidth]{figs/trieste_plots/fit2}
        \end{center}
        \caption{Results of fit to the $W + bb$ component with $t\bar{t}$ constrained to the
          results in the multi-jet control region. Electron (left) and
          muon (right) decay channel.}
        \label{fig:intermediatefit1}
\end{figure}

\subsection{Top-antitop and W + bb correlated fit}
Subsequently, a fit has been performed to the
transverse mass distribution for the $t\bar{t}$ and $W + bb$
samples, with the $t\bar{t}$ normalization constrained to the
measured cross section reported in table~\ref{tab:mcxsecs}.
The $3.5\%$ uncertainty on the $t\bar{t}$ cross section has been
adopted to constrain the $t\bar{t}$ sample.
An additional degree of freedom has been added to the fit by
including a common $c_{\mathrm{scale}}$ correlating the $t\bar{t}$ and
$c_{W+bb}$ samples and has been constrained to the
values of $c_{t\bar{t}}^{\mathrm{multi-jet}}$ obtained in the
$t\bar{t}$ multi-jet control region:
$$c_{\mathrm{scale}}(W \rightarrow e \nu + bb) = 1.141 \pm 0.016 \mathrm{,}$$
$$c_{\mathrm{scale}}(W \rightarrow \mu \nu + bb) = 1.118 \pm 0.013 \mathrm{,}$$
As expected, the result of the fit for the $c_{t\bar{t}}$
scale factors is consistent with unity:
$$c_{t\bar{t}}(W \rightarrow e \nu + bb) = 0.991 \pm 0.030 \mathrm{,}$$
$$c_{t\bar{t}}(W \rightarrow \mu \nu + bb) = 1.006 \pm 0.029 \mathrm{.}$$
The signal $W+bb$ scale factor has been modified accordingly:
$$c_{W+bb}(W \rightarrow e \nu + bb) = 1.501 \pm 0.154 \mathrm{,}$$
$$c_{W+bb}(W \rightarrow \mu \nu + bb) = 1.432 \pm 0.124 \mathrm{.}$$
The results are graphically represented in Figure~\ref{fig:intermediatefit2}.
\begin{figure}[htb]
        \begin{center}
                \leavevmode
                \includegraphics[width=0.95\textwidth]{figs/trieste_plots/fit3}
        \end{center}
        \caption{Results of fit to the $t\bar{t}$ and $W+bb$ components correlated by
          a common scale factor. The $t\bar{t}$ is constrained to the results
          in the multi-jet control region. Electron (left) and
          muon (right) channel.}
        \label{fig:intermediatefit2}
\end{figure}
The results of this fit are, in practice, equivalent to the previous results
with the only difference in the fact that the modification of the $t\bar{t}$
contribution is delegated to the $c_{\mathrm{scale}}$ scale parameter that
is in common with the signal sample; as a consequence, the fit converges to a
smaller value of $c_{W+bb}$, which however does not affect the final cross section
measurement.
Indeed, this step aims at demonstrating a correlated general underestimation of data
in simulation. Such a correlation is compatible with a systematic effect in the calibration
and selection of the objects in the selected final state, common to all simulated
samples; the jet energy corrections and the $b$-tagging efficiency scale factors
may provide an explanation for this effect, since they have a big associated
systematic uncertainty which will be described in the next chapter.


\subsection{Baseline results}
In this paragraph the fitting procedure adopted to estimate the
cross section is described.
In the same fashion as for the previous configuration,
the $t\bar{t}$ normalization is constrained to its experimental
value and uncertainty.
The $c_{\mathrm{scale}}$ global scale is constrained to the
results in the $t\bar{t}$ multi-jet control region:
$$c_{\mathrm{scale}}(W \rightarrow e \nu + bb) = 1.141 \pm 0.016 \mathrm{,}$$
$$c_{\mathrm{scale}}(W \rightarrow \mu \nu + bb) = 1.118 \pm 0.013 \mathrm{,}$$
and it is now correlating the
variation of all the simulated samples with the only exception
of the QCD sample: top-antitop, Single~top, the light and $c$ fractions
of the $W+$jets sample, the Drell-Yan sample and the di-boson samples.
This modification has the effect of further reducing the
$W+bb$ signal strength:
$$c_{W+bb}(W \rightarrow e \nu + bb) = 1.419 \pm 0.173 \mathrm{,}$$
$$c_{W+bb}(W \rightarrow \mu \nu + bb) = 1.351 \pm 0.135 \mathrm{,}$$
The results are graphically represented in Figure~\ref{fig:baselinefit}.
\begin{figure}[htb]
        \begin{center}
                \leavevmode
                \includegraphics[width=0.8\textwidth]{figs/trieste_plots/electrons/w_mt_wenu_bb_wide_doFit}
                \includegraphics[width=0.8\textwidth]{figs/trieste_plots/muons/w_mt_wmnu_bb_wide_doFit}
        \end{center}
        \caption{Results of fit to the $t\bar{t}$, $W+bb$, $W+$light, Single~$t$ and
          di-boson samples, correlated by
          a common scale factor. The $t\bar{t}$ is constrained to the results
          in the multi-jet control region. Electron (top) and
          muon (bottom) channel.}
        \label{fig:baselinefit}
\end{figure}
The effects of the normalization scale factors obtained on the fit
on the number of selected events is reported in table~\ref{tab:yieldsPostfit},
to be compared with the pre-fit event yields reported in
table~\ref{tab:yieldsPrefit}.
\begin{table}[htb]
\begin{center}
\begin{tabular}{|c|c|c|}
\hline
\textbf{Sample} & $W \rightarrow e \nu + bb$ & $W \rightarrow \mu \nu + bb$ \\
\hline
DATA        & 5073   & 5322 \\
\hline
$W + bb$    & 1182.2 & 1316.7 \\
$W + c$     & 78.3   & 82.7 \\
$W + light$ & 153.5  & 154.4 \\
$W + (\rightarrow \tau \nu) + bb$  & 17.5 & 25.4 \\
$t\bar{t}$  & 2315.1 & 2566.4 \\
Single $t$  & 667.5  & 783.8 \\
$VV$        & 106.0  & 126.8 \\
$Z + jets$  & 96.1   & 144.7 \\
\hline
\end{tabular}
\end{center}
\caption{Selected event yields after the baseline fitting procedure described in
  section~\ref{sec:transversemassfit}.}
\label{tab:yieldsPostfit}
\end{table}


