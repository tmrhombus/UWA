\section{Results and comparison with theoretical prediction}\label{Results}

OK this is still a little ways out

%The $\Wbb$ cross section within the reference fiducial cuts
%$\pt^{\mu}>25$ GeV, $|\eta^\mu|<2.1$, $\pt^{jets}>25$ GeV, $|\eta^{jets}|<2.4$
%is determined using the following Equation:
% 
%\begin{eqnarray}
%      {\displaystyle \sigma(\Wbb) = \frac{N_{Data}-N_{Bckg}}{\Lint~\epsilon_{sel}}}
%\nonumber
%\end{eqnarray}
% 
%\noindent
%where $N_{sel}$ is the number of selected events after subtraction of the
%background, determined through the fit procedure described above,
%$\Lint=5.0~fb^{-1}$ is the integrated luminosity and $\epsilon_{sel}$ is the final selection
%efficiency of our process.
%
%The observed number of events in data after selection in the signal region is $N_(S+B)_{data}=1110\pm33$. 
%After background substraction via fit, we observe a number of signal events equal to $N_{sel}=267\pm50$, as
%detailed in the previous section.
%
%The  $\epsilon_{sel}$ (equivalent to $~\AccEff$) required to unfold our results to the b-hadron level 
%is measured in Madgraph, and found to be $5.1\pm0.01 \textrm{PDF}) \pm0.5(Q^2,\textrm{matching})\%$. 
%Note that it is refered to the fiducial cuts, and not extrapolated to the full $\mu$ phase-space.  
%
%The cross section is therefore measured to be:
%
%\begin{eqnarray}
%% {\displaystyle \sigma(\Wbb, \pt^\mu>25~\GeV) = 1.35 \pm 0.24~(stat.) \pm XXX ~(exp.)  \pm 0.09 (theo.) pb.  }
%{\displaystyle \sigma(\Wbb, \pt^\mu>25~\GeV, |\eta^{\mu}|<2.1, \pt^{b~jets}>25~\GeV,  |\eta^{b~jets}|<2.4) = \\ 
%\nonumber
%%                              1.41 \pm 0.12~(stat.)~^{-0.24}_{+0.25}~(exp.) \pm 0.09 (theo.) pb.}
%1.17\pm  0.10\stat ~^{+0.21}_{-0.17}\syst \pm 0.18 \theo \pm 0.03\lumi \unit{pb.}
%\nonumber
%\end{eqnarray}
%
%Alternatively we can unfold our measurements to the bhadron level, already including the jet vetos at the gen level. 
%In this case, the acceptance is found to be $11.25\pm0.01 \textrm{PDF}) \pm1.0(Q^2,\textrm{matching})\%$ and cross-section is measured to be:
%\begin{eqnarray}
%% {\displaystyle \sigma(\Wbb, \pt^\mu>25~\GeV) = 1.35 \pm 0.24~(stat.) \pm XXX ~(exp.)  \pm 0.09 (theo.) pb.  }
%{\displaystyle \sigma(\Wbb, \pt^\mu>25~\GeV, |\eta^{\mu}|<2.1, \pt^{b~jets}>25~\GeV,  |\eta^{b~jets}|<2.4) = \\ 
%\nonumber
%%                              1.41 \pm 0.12~(stat.)~^{-0.24}_{+0.25}~(exp.) \pm 0.09 (theo.) pb.}
%0.53\pm  0.05\stat ^{+0.09}_{-0.08}\syst \pm 0.06 \theo \pm 0.01\lumi \unit{pb.}
%\nonumber
%\end{eqnarray}
%
%
%The theoretical prediction for the cross-section, computed using the MCFM~\cite{MCFM} program at NLO
%is found to be in agreement with our measurement at one sigma level.
%
%%% Add here a line about the comp with ATLAS?
%
%\begin{table}[htb]
%\begin{center}
%\begin{tabular}{|c|c|c|}
%\hline
%MCFM                     & \multicolumn{2}{|c|}{\MADGRAPH+\PYTHIA} \\     \hline
%MSTW08NNLO      &    Final partons                          &  B-Hadrons   \\ \hline
%(kt dR=0.5)              &  (tagged anti-kt dR=0.5 partonJets)       &  (tagged anti-kt dr=0.5 particle Jets) \\ \hline  
%$0.52\pm{0.03}$ pb       &    $0.65\pm0.02$(stat) pb                  &   $0.59\pm0.02$(stat) pb   \\ \hline 
%\multicolumn{3}{c}{~}\\
%\hline 
%\multicolumn{3}{|c|}{Fit Result (signal strength), compared to MG+Pythia}   \\ \hline
%\multicolumn{3}{|c|}{$\rho_{WBB} = 0.90    \pm0.08 \textrm{(stat.)}   ~^{-0.13}_{+0.16} \textrm{(sys.)}$}  \\ \hline 
%\multicolumn{3}{c}{~}\\
%\hline
%%\multicolumn{3}{|c|}{Experimental Measurement (unfolded to MG+Pythia)} \\ \hline
%%\multicolumn{3}{|c|}{$1.09\pm 0.10~\textrm{(stat.)} \pm 0.18~\textrm{(exp.)} \pm 0.08 \textrm{(pdf.)} \pm 0.02 \textrm{(lumi.)}$ pb } \\ \hline 
%%\multicolumn{3}{c}{~}\\
%%\hline
%%\multicolumn{3}{|c|}{Experimental Measurement (unfolded to Hard Partons)} \\ \hline
%%\multicolumn{3}{|c|}{$0.97 \pm 0.09~\textrm{(stat.)} \pm 0.16~\textrm{(exp.)} \pm 0.08 \textrm{(pdf.)} \pm 0.02 \textrm{(lumi.)}$ pb } \\ \hline 
%%\hline
%\multicolumn{3}{|c|}{Experimental Measurement (unfolded to B-Hadrons, Jet Vetos at Gen Level)} \\ \hline
%\multicolumn{3}{|c|}{$0.53\pm  0.05\stat \pm 0.09 \syst \pm 0.06 \theo \pm 0.01\lumi \unit{pb}$} \\ \hline  
%\end{tabular}
%\caption{Comparison of the measured Wbb cross-section and theoretical references computed using MCFM
%and the \MADGRAPH+\PYTHIA reference Monte Carlo prediction.}
%\label{tab:comparisonMCFM_V2}
%\end{center}
%\end{table}
%
%
%\begin{table}[htb]
%\begin{center}
%\begin{tabular}{|c|c|c|}
%\hline
%MCFM                             & \multicolumn{2}{|c|}{\MADGRAPH+\PYTHIA} \\     \hline
%MSTW08NNLO                       &    Final partons                          &  B-Hadrons   \\ \hline
%(kt dR=0.5)                      &  (tagged anti-kt dR=0.5 partonJets)       &  (tagged anti-kt dR=0.5 particleJets) \\ \hline
%$0.94 ^{+0.20}_{-0.15}$ pb       &    $1.48\pm0.02$(stat) pb                  &   $1.30\pm0.02$(stat) pb   \\ \hline
%\multicolumn{3}{c}{~}\\
%\hline
%\multicolumn{3}{|c|}{Fit Result (signal strength), compared to MG+Pythia}   \\ \hline
%\multicolumn{3}{|c|}{$\rho_{WBB} = 0.90    \pm0.08 \textrm{(stat.)}   ~^{-0.13}_{+0.16} \textrm{(sys.)}$}  \\ \hline
%\multicolumn{3}{c}{~}\\
%\hline
%%\multicolumn{3}{|c|}{Experimental Measurement (unfolded to MG+Pythia)} \\ \hline
%%\multicolumn{3}{|c|}{$1.09\pm 0.10~\textrm{(stat.)} \pm 0.18~\textrm{(exp.)} \pm 0.08 \textrm{(pdf.)} \pm 0.02 \textrm{(lumi.)}$ pb } \\ \hline 
%%\multicolumn{3}{c}{~}\\
%%\hline
%%\multicolumn{3}{|c|}{Experimental Measurement (unfolded to Hard Partons)} \\ \hline
%%\multicolumn{3}{|c|}{$0.97 \pm 0.09~\textrm{(stat.)} \pm 0.16~\textrm{(exp.)} \pm 0.08 \textrm{(pdf.)} \pm 0.02 \textrm{(lumi.)}$ pb } \\ \hline 
%%\hline
%\multicolumn{3}{|c|}{Experimental Measurement (unfolded to B-Hadrons)} \\ \hline
%\multicolumn{3}{|c|}{$1.17\pm  0.10\stat ^{+0.21}_{-0.17}\syst \pm 0.18 \theo \pm 0.03\lumi \unit{pb}$} \\ \hline 
%\end{tabular}
%\caption{Comparison of the measured Wbb cross-section and theoretical references computed using MCFM
%and the \MADGRAPH+\PYTHIA reference Monte Carlo prediction. Vetos in the Acceptance.}
%\label{tab:comparisonMCFM_V3}
%\end{center}
%\end{table}
%
%
%%\begin{table}[htb]
%%\begin{center}
%%\begin{tabular}{|c|c|c|c|}
%%\hline
%%$p_T^\mu$ &Data& Fit Result  &  Experimental Measurement         \\ 
%%(GeV)& (evts)   &     $\rho_{WBB}$       &  (unfolded to MG+Pythia)    \\ \hline
%%
%%25 & 1230  & $0.90    \pm0.08~\textrm{(st)}   ~^{-0.13}_{+0.16}~\textrm{(sys)}$ & 
%%                         $1.09 \pm 0.10~\textrm{(st)}\pm 0.18~\textrm{(exp)}\pm 0.08~\textrm{(pdf)} \pm 0.02~\textrm{(lum)}$ pb  \\
%%\hline
%%30 & 1100  & $0.89    \pm0.10~\textrm{(st)}   ~^{-0.17}_{+0.19}~\textrm{(sys)}$ &
%%                         $1.09\pm 0.12~\textrm{(exp} \pm 0.22~\textrm{(exp)} \pm 0.08~\textrm{(pdf)} \pm 0.02~\textrm{(lum)}$ pb\\
%%\hline
%%\end{tabular}
%%\caption{Comparison of results with a muon $p_T$ cut of 25 GeV and of 30 GeV.}
%%\label{tab:comparison_V3}
%%\end{center}
%%\end{table}
%%
%
%%0.893-0.2/+0.22 
%% +/-0.11
%
%
%% Commented out because I do not understand them. To be checked.
%
%%The post fit scale factor for $\Wbb$ is $0.803^{+0.146}_{-0.153} (syst.) \pm XXX (stat)$.
%%\begin{table}[htb]
%%\begin{center}
%%\small
%%\begin{tabular}{|c|c|c|c|}
%%\hline
%
%%Nuisance & Post-Fit Scale Factor & Error & Global Correlation\\
%%
%%\hline
%%\hline
%%W+light    &0.898&$\pm 0.957$&0.18\\
%%W+C        &0.878&$\pm 0.713$&0.61\\
%%t\bar{t}   &0.991&$\pm-0.14$&0.969\\
%%Single Top &1.020&$\pm0.996$&0.197\\
%%tW         &1.01&$\pm0.997$&0.075\\
%%Z+Jets     &0.98&$\pm0.987$&0.140\\
%%WZ         &1.01&$\pm0.996$&0.028\\
%%JES        &-0.681&$\pm 0.174 \sigma$&0.459\\
%%UCE        &-0.681&$\pm 0.323 \sigma$&0.234\\
%%MES        &-0.658&$\pm 0.599 \sigma$&0.207\\
%%\hline
%%\end{tabular}
%%\caption{Post Fit Effects on Nuisance Parameters.}
%%\label{tab:postfitNiu}
%%\end{center}
%%\end{table}
%%                  beff    0.0000e+00   -4.4826e-02         +/-  9.53e-01  0.961326
%%                 beffC    0.0000e+00   -1.5375e-01         +/-  9.77e-01  0.333913
%%                 beffT    0.0000e+00    4.1464e-02         +/-  9.84e-01  0.877986
%%                   bgT    0.0000e+00    9.8701e-02         +/-  9.96e-01  0.197488
%%                 bgTOP    0.0000e+00   -1.4035e-01         +/-  3.70e-01  0.969309
%%                bgTbar    0.0000e+00    7.3306e-02         +/-  9.97e-01  0.092798
%%                   bgW    0.0000e+00   -3.1395e-01         +/-  9.57e-01  0.182314
%%                  bgWZ    0.0000e+00   -5.6908e-03         +/-  9.96e-01  0.028613
%%                  bgWc    0.0000e+00   -5.1859e-01         +/-  7.13e-01  0.612337
%%                bgZjet    0.0000e+00   -8.4700e-02         +/-  9.87e-01  0.140418
%%                  bgtW    0.0000e+00    6.9489e-02         +/-  9.97e-01  0.074883
%%                   jet    0.0000e+00   -5.8904e-01         +/-  1.74e-01  0.459612
%%                jetUCE    0.0000e+00    6.5969e-01         +/-  3.29e-01  0.233670
%%                  lumi    0.0000e+00    1.0918e-02         +/-  9.88e-01  0.838966
%%                  muon    0.0000e+00    4.6307e-01         +/-  4.06e-01  0.207431
%%                   qcd    0.0000e+00    1.1045e+00         +/-  3.20e-01  0.454662
%%                     r    1.0000e+00    8.0301e-01 (+1.52e-01,-1.45e-01)  0.768408
%
%%Uncertainties
%%were calculated using the Correlation matrix as see in Table XXX. The post-fit scale factor and error for 
%e%ach nuisance parameter is seen in Table (\ref{tab:postfitNiu}). Here the global correlation for nuisance parameter \textit{n}
%%is a number between zero and one which is the linear combination of all other parameters which are most strongly correlated
%%with \textit{n}. \ref{MNUserGuide}
%
%
%
%
%%% Moved earlier
%%\begin{figure}[htbp]
%%    \center
%%    \subfigure[]{\label{fig:finalJ1SVMass}\includegraphics[width=0.3\textwidth]{fig/final/J1SVMassb.pdf}}
%%    \subfigure[]{\label{fig:finalJ2SVMass}\includegraphics[width=0.3\textwidth]{fig/final/J2SVMassb.pdf}}
%%    \subfigure[]{\label{fig:finalMET}\includegraphics[width=0.3\textwidth]{fig/final/met_Add3.pdf}} \\
%%    \subfigure[]{\label{fig:finalMtCal}\includegraphics[width=0.3\textwidth]{fig/final/MtCal_Add3.pdf}}
%%    \subfigure[]{\label{fig:finalWPt}\includegraphics[width=0.3\textwidth]{fig/final/WPt.pdf}}
%%    \subfigure[]{\label{fig:finalWPt}\includegraphics[width=0.3\textwidth]{fig/final/ht.pdf}}\\
%%% there is some problem with these plots:
%%%    \subfigure[]{\label{fig:finalMjj}\includegraphics[width=0.25\textwidth]{fig/final/mJJ.pdf}}
%%%    \subfigure[]{\label{fig:finalPtJJ}\includegraphics[width=0.25\textwidth]{fig/final/ptJJ.pdf}}
%%      \label{fig:postfitdistributions}
%%    \caption{Post Fit Distributions.}
%%\end{figure}
%%
%
%
%
%
%%\section{Conclusions}\label{Conclusions}
%%List Cuts, Acceptances, Final Cross Section
%%Further studies needed for individual dcay modes
